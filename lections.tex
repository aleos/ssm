%% lections.tex

% Выбор класса документа
\documentclass[
a4paper,
12pt,
draft
]{article}

\usepackage%[warn]%
{mathtext} % русские буквы в формулах
\usepackage{textcomp} % Дополнительный набор текстовых символов

%% "Висячие" переносы
\defaulthyphenchar=127
\protected\def\-{\ifnum\hyphenchar\font>0\discretionary{\char\hyphenchar\font}{}{}\fi} 

\usepackage{cmap}
\usepackage[T2A]{fontenc}   % Выбор внутренней ТеХ-кодировки
\usepackage[utf8]{inputenc} % Выбор кодовой страницы документа
\usepackage[english, russian]{babel} % Выбор языка документа

\DeclareSymbolFont{T2Aletters}{T2A}{cmr}{m}{it} % курсивные математические русские буквы

\usepackage{
  indentfirst, % Первый параграф раздела с красной строки
  fullpage % Более полное заполнение листа (texlive-latex-extra)
}

\binoppenalty=10000 % Разрывы строк после знаков бинарных операций
\relpenalty=10000   % Разрывы строк после знаков бинарных отношений
\usepackage{
  amsmath, % \text
  amssymb,  % \varnothing \leqslant \geqslant
  amsthm   % Теоремы, определения, замечания.
}

\usepackage{stackrel} % символы под знаком (расширение команды \stackrel)

\usepackage{qtree} % Синтаксические деревья
\newcommand{\qleafhook}[1]{\textbf{#1}} % Листья деревьев выделены \textbf


\usepackage[labelsep=period]{caption} %заменить умолчальное разделение
                                %':' на '.' в подписях к рисункам и
                                %таблицам

\usepackage{multirow}  % Объединение строк в таблице
\usepackage{longtable} % Перенос таблиц между страницами
\usepackage{array}     % 

%\usepackage[matrix,arrow,curve]{xy} % Коммутативные диаграммы с
                                % помощью Xy-pic
\usepackage[all]{xy}


\theoremstyle{definition}
\newtheorem{defin}{Определение}
\theoremstyle{remark}
\newtheorem*{ex}{Пример}
\newtheorem*{rem}{Замечание}
\theoremstyle{plain}
\newtheorem*{theorem}{Теорема}
\newtheorem*{consequence}{Следствие} %consequence of the theorem

\newcounter{N}

\title{Структурно-стохастические методы}
%\author{Михаил Юрьевич Охтилев}
\date{}

\includeonly{
ch5,
ch6,
}

\begin{document}

\maketitle
\tableofcontents
\setcounter{section}{5}
\subsection{Основные понятия теории формальных грамматик и языков}

\subsubsection{Определение формальной грамматики (ФГ) и языка}


\emph{Теория формальных грамматик и языков} является основным разделом
математической лингвистики~--- математической дисциплины,
ориентированной на изучение структуры естественных и искусственных
языков.

Эта теория возникла в 50-е годы в работах американского лингвиста
Хомского. Хомский исходил из потребностей естественного языка. Однако,
вскоре стало ясно, что методы его теории в не меньшей степени
применимы и к естественным языкам (в частности, информационный язык,
формируемый системой сбора ТМИ).

По характеру используемого математического аппарата теория формальных
грамматик и языков близка к теории алгоритмов и к теории автоматов.

Основным понятием в теории формальных грамматик является понятие
формальной грамматики.

\begin{defin}
  \emph{Формальной грамматикой} $G$ называется кортеж (четвёрка)
  \begin{equation}
    \label{eq:G}
    G = \left<T, N, R, S\right>,
  \end{equation}
  
  {\hangindent=4em
    {\noindent где $T$~--- конечное множество терминальных символов, алфавит
      терминальных символов, терминалов;}
    
    $N$~--- конечное множество нетерминальных (вспомогательных)
    символов; $T \cap N = \varnothing$
    
    $R$~--- конечное множество упорядоченных пар $(\alpha, \beta)$,
    имеющих вид $\alpha \to \beta$, и называемых правилами подстановки,
    вывода, продукциями грамматики.
    
    $S \in N$~--- начальный символ (аксиома).
  }
\end{defin}

\begin{rem}
  Если $T$~--- терминальный алфавит, то $T^{*}$~--- множество слов
  в алфавите~$T$.
\end{rem}

\begin{rem}
  Будем в дальнейшем обозначать:
  \begin{itemize}
  \item элементы алфавита $T$: $a, b, c, \ldots$
  \item элементы алфавита $N$: $A, B, C, \ldots$
  \end{itemize}
\end{rem}

\begin{defin}
  Продукции грамматики $G$ определим как:
  \begin{equation}
    \label{eq:rule}
    R = \{\alpha\to\beta \mid \alpha \in (N \cup T)^{*} \times N \times (N
    \cup T)^{*},\; \beta \in (N \cup T)^{*}\}
  \end{equation}
\end{defin}

\begin{rem}
  Свойства отношения "$\to$" (отношения порядка):
      \begin{itemize}
      \item рефлексивность,
      \item антисимметричность,
      \item транзитивность.
      \end{itemize}
\end{rem}

\begin{defin}[выводимости слов]
  Будем говорить, что
  \begin{equation}
    \label{eq:deduction}
    \gamma \stackrel[G]{}{\Rightarrow} \varepsilon\qquad (\gamma,
    \varepsilon \in (N \cup T)^*)\,,
  \end{equation}
  если $\gamma = \sigma_1 \alpha \sigma_2$, $\varepsilon = \sigma_1
  \beta \sigma_2$ ($\sigma_1, \sigma_2 \in (N \cup T)^*$),
  и если подстановка $\alpha \to \beta$ входит в схему~$R$ грамматики~$G$.
\end{defin}

\emph{Содержательно} $\gamma \Rightarrow \varepsilon$ означает, что
цепочка $\varepsilon$ выводима из цепочки $\gamma$ путём подстановки в
цепочку $\gamma$ вместо цепочки $\alpha$ цепочки $\beta$, причём
подстановка входит в множество $R$: $$\alpha \to \beta\,.$$

\begin{defin}
  Будем говорить, что $$\gamma \stackrel[G]{*}{\Rightarrow}
  \varepsilon \; (\gamma,\varepsilon \in (N \cup T)^{*})\,,$$
  если существует $\gamma_0 = \gamma$ и $\gamma_n = \varepsilon$,
  такие, что $$\gamma_0 \Rightarrow \gamma_1 \Rightarrow \gamma_2
  \Rightarrow \ldots \Rightarrow \gamma_{n-1} \Rightarrow \gamma_n\,.$$
\end{defin}

Последовательность $\gamma_0, \gamma_1, \ldots, \gamma_n$ называется
\emph{выводом} длины $n$.

В ФС: \emph{выводимость} (непосредственная выводимость): $\alpha
\vdash \beta$ ($\frac{\phantom{0}\alpha\phantom{0}}{\beta}$)~--- правило
вывода, а выводимость $[\alpha \Rightarrow \beta] \Leftrightarrow
[\alpha = \varepsilon_1 \vdash \varepsilon_2 \vdash \varepsilon_3
\vdash \ldots \vdash \varepsilon_n = \beta]$

\begin{defin}
  \emph{Языком} $L(G)$, порождённым грамматикой $G$, называется
  следующее подмножество множества $T^*$:
  \begin{equation}
    \label{eq:lang}
    L(G) = \{\alpha \mid S \stackrel[G]{*}{\Rightarrow} \alpha,\;
    \alpha \in T^*\}\,,
  \end{equation}
  т.\,е. множество всех слов $\alpha \in T^*$, состоящих только из
  терминальных символов, выводимо в грамматике из начальной аксиомы $S$.
\end{defin}

\begin{defin}
  Количество символов, входящих в слово $\alpha \in L(G)$, будем
  называть его \emph{длиной} и обозначать $|\alpha|$.
\end{defin}

\begin{ex}
  \label{ex:G}
  $G = \langle T,N,P,S\rangle$, где $T=\{a,+,*\}$, $N = \{S,A\}$,
  $P$:
  \begin{enumerate}
  \item $S \to S*S$;
  \item $S \to A+A$;
  \item $A \to A*A$;
  \item $A \to a$.
  \end{enumerate}
  Вывод:
  $S \stackrel{2}{\Rightarrow} A+A \stackrel{4}{\Rightarrow} A*A+A
  \stackrel{5}{\Rightarrow} a*A+A \stackrel{5}{\Rightarrow} a*a+A
  \stackrel{5}{\Rightarrow} a*a+a$.

Язык $L(G)$~--- множество правильно построенных форм (ППФ) с
использованием операций $+$, $*$ и переменной $a$.
\end{ex}

\begin{defin}
  Помеченное дерево $D$ называется \emph{деревом (графом) вывода}
  грамматики $G$, если:
  \begin{enumerate}
  \item Корень дерева помечен символом аксиомы $S$;
  \item Если $D_1,D_2$~--- поддеревья дерева $D$, то корень каждого
    поддерева $D_i$ помечен:
    \begin{itemize}
    \item символом $A_i\;(A_i \in N)$, если дерево $D_i$ имеет больше
      одной вершины;
    \item символом $X_i\;(X_i \in T)$, если дерево $D_i$ состоит из
      единственной вершины.
    \end{itemize}
  \end{enumerate}
\end{defin}

\begin{rem}
  $D_i$ является деревом вывода в грамматике $G_i = \langle
  T,N,D,X_i\rangle$ (здесь $X_i$~--- есть аксиома дерева $D_i$).
\end{rem}

\begin{ex}
  Для грамматики $G$ (из прошлого примера на стр. \pageref{ex:G})
  вывод цепочки $a*a+a$ представлен на рис.~\ref{fig:deduction}.
  
  \begin{figure}[h]

    \Tree [.S [.A [.A a ] * [.A a ] ] + [.A a ] ]
    
    \caption{Вывод цепочки $a*a+a$}
    \label{fig:deduction}
  \end{figure}
  
\end{ex}

\begin{defin}
  Грамматика называется \emph{однозначной}, если каждое слово языка
  $L(G)$ имеет только одно дерево вывода. В противном случае
  грамматика называется \emph{неоднозначной}.
\end{defin}

\begin{rem}
  Свойство однозначности характеризует \emph{грамматику}, а не язык,
  поскольку один и тот же язык может быть описан различными
  грамматиками, среди которых могут быть как однозначные, так и
  неоднозначные.
\end{rem}

\begin{ex}
  Для той же грамматики (см. пример на с.~\pageref{ex:G}) и цепочки
  $a*a+a$:

  \begin{figure}[h]
    \centering
    \qtreecenterfalse

    (1)
    \Tree [.S [.A [.A a ] * [.A a ] ] + [.A a ] ]
    \hskip .1\textwidth
    (2)
    \Tree [.S\1 [.S [.A a ] ] * [.S [.A a ] + [.A a ] ] ]

    \caption{Два варианта вывода цепочки $a*a+a$}
  \end{figure}
\end{ex}

%%%%%%%%%%%%%%%%%%%%%%%%%%%%%%%%%%%%%%%%%%%%%%%%%%%%%%%%%%%%%%%%%%%%%%%%%%%%%%%%

\subsubsection{Классификация формальных грамматик и языков}

Формальные грамматики (ФГ) делятся на 3 категории:
\begin{itemize}
\item распознающие ФГ;
\item порождающие ФГ;
\item преобразующие ФГ.
\end{itemize}

\begin{defin}
  ФГ называется \emph{распознающей}, если для любой рассматриваемой
  цепочки она позволяет ответить на вопрос: является ли эта цепочка
  правильной ($\in L(G)$) или нет, и в случае положительного ответа
  даёт описание структуры (строение) этой цепочки.

  ФГ называется \emph{порождающей}, если она позволяет строить любую
  правильную цепочку, давая при этом описание её структуры, и не
  позволяет строить ни одной неправильной цепочки.

  ФГ называется \emph{преобразующей}, если для любой правильно
  построенной цепочки она позволяет строить \emph{отображение} её в
  виде цепочки, задавая при этом порядок реализации этого отображения.
\end{defin}

Рассмотрим класс \emph{порождающих} грамматик.

Основатель теории ФГ Н.\,Хомский провёл следующую классификацию (по
виду правил):


\begin{table}[!hp]
  \centering
  \begin{tabular}[!hp]{|c|p{.39\textwidth}|p{.4\textwidth}|}
    \hline
    Тип ФГ & Название ФГ & Вид порождающих правил ${P = \{\alpha \to
    \beta\}}$ \\
    \hline
    0 & Без ограничений & $\alpha \in (T \cup N)^* \times N \times (T
    \cup N)^*$, ${\beta \in (T \cup N)^*}$\\
    \hline
    1 & Контекстно-зависимые (непосредственно составляющих,
    неукорачивающие) & $\alpha \in (T \cup N)^* \times N \times (T
    \cup N)^*$, ${\beta \in (T \cup N)^*}$, $|\alpha| \leqslant
    |\beta|$\\
    \hline
    2 & Контекстно-свободные & $\alpha \in N$, $\beta \in (T \cup
    N)^*$ \\
    \hline
    3 & Регулярные (праволинейные) & $\alpha \in N$, $\beta \in T
    \times N$, $\beta \in T$\\
    \hline
  \end{tabular}
  \label{tab:chomsky}
\end{table}

\begin{ex}
  Правила вывода для разных типов языков
  \begin{center}
    \begin{tabular}[!hp]{p{11em}p{11em}p{8em}}
      $S \to aB$ & $S \to ABa$  & $AS \to ABC$ \\
      $B \to bC$ & $A \to cAAc$ & $Ab \to bbCb$\\
      $C \to b$  & $B \to a$    & $B  \to b$   \\
                 &              & $C  \to c$   \\[.5em]
      Тип 3 (Автоматная) & Тип 2 (КС) & Тип 1 (НС)\\
    \end{tabular}
  \end{center}
\end{ex}

\begin{rem}
  ФГ является одним из видов формальной системы с:
  \begin{enumerate}
  \item алфавитом $A = T \cup N$ (или языком $L(G)$);
  \item аксиомой $S \in N$;
  \item правилами вывода $P = \{\alpha \to \beta\}$.
  \end{enumerate}
\end{rem}

\begin{defin}
  \emph{ФГ (ФЯ) типа 0}~--- такие ФГ, в которых не накладывается
  никаких ограничений на правила подстановок. Эти ФГ позволяют
  порождать любые \emph{рекурсивно-перечислимые множества} и
  эквивалентны по мощности МТ, частично-рекурсивным функциям.
\end{defin}

\begin{defin}
  \emph{ФГ (ФЯ) типа 1} или ФГ НС, или КЗ, или \emph{неукорачивающие
    ФГ}~--- такие ФГ, в которых каждое правило $\alpha \to \beta$
  удовлетворяет соотношению $|\alpha| \leqslant |\beta|$ и каждое
  правило подстановки имеет вид (или может быть приведено к виду)
  \begin{equation*}
    \sigma_1 A \sigma_2 \to \sigma_1 \gamma \sigma_2\,,\quad A \in N,\;\;
    \sigma_1,\sigma_2 \in (T \cup N)^*,\;\; \gamma \in (T \cup N)^+
  \end{equation*}
  ФЯ типа 1~--- \emph{рекурсивное множество}.
\end{defin}

\begin{rem}
  В ФГ типа 1 могут быть правила $AB \to BA$
  ($\boxed{\phantom{A}}$~--- контексты, ${A',B' \in N}$~--- новые
  нетерминалы):
  \begin{equation*}
    AB \to BA \Leftrightarrow A\boxed{B} \to A'\boxed{B}
    \Leftrightarrow \boxed{A'}B \to \boxed{A'}B' \Leftrightarrow
    A'\boxed{B'} \to B\boxed{B'} \Leftrightarrow \boxed{B}B' \to
    \boxed{B}A\ldotp
  \end{equation*}

\end{rem}

\begin{defin}
  \emph{ФГ (ФЯ) типа 2}~или КС-грамматики~--- такие ФГ, в которых
  каждое правило подстановки имеет вид
  \begin{equation*}
    A \to \gamma,\quad \text{где} A \in N,\;\; \gamma \in (N \cup T)^*\,.
  \end{equation*}
  ФЯ типа 2~--- \emph{рекурсивное множество}.
\end{defin}

\begin{defin}
  \emph{ФГ (ФЯ) типа 3} (регулярные, автоматные)~--- такие ФГ, в
  которых каждое правило имеет следующий вид:
  $$
  \begin{cases}
    A \to aB & \text{или} \\
    A \to a, & \text{где $A,B \in N$, $a \in T$}\\
  \end{cases}
  $$
  ФЯ типа 3~--- \emph{регулярные события}.
\end{defin}

%%%%%%%%%%%%%%%%%%%%%%%%%%%%%%%%%%%%%%%%%%%%%%%%%%%%%%%%%%%%%%%%%%%%%%%%%%%%%%%%

\subsubsection{Формальные свойства грамматик}

При решении задач, связанных с грамматиками (с распознаванием слов по
грамматикам, с построением грамматик, с преобразованием грамматик,
\ldots) мы всегда сталкиваемся с проблемами их разрешимости: то есть
существует ли вообще алгоритм, позволяющий решать ту или иную
задачу. В этом случае нужно обратиться к изучению \emph{формальных
  свойств грамматик}, или, к \emph{алгоритмическим проблемам теории
  формальных грамматик}. Перечислим некоторые из них ($+$~---
разрешимая проблема, $-$~--- неразрешимая):


\begin{longtable}{|p{.6\textwidth}|p{.05\textwidth}|p{.05\textwidth}|p{.05\textwidth}|p{.05\textwidth}|}
  \hline \multirow{2}*{Название проблемы} & \multicolumn{4}{c|}{Тип
    грамматики}\\
  \cline{2-5}
  & 3 & 2 & 1 & 0\\
  \hline \endhead 1. Пуст ли язык, порождённый данной грамматикой
  ($L_G =
  \varnothing$)? & $+$ & $+$ & $-$ & $-$ \\
  \hline 2. Бесконечен ли язык, порождённый данной грамматикой ($|L_G|
  =
  \infty$)? & $+$ & $+$ & $-$ & $-$ \\
  \hline 3. Включает ли язык, порождённый данной грамматикой, все
  слова в
  алфавите $T$ ($L_G = T^*$)? & $+$ & $-$ & $-$ & $-$ \\
  \hline 4. Составляет ли язык, порождённый данной грамматикой,
  подмножество языка, порождаемого другой ($L_{G1} \subseteq
  L_{G2}$)? & $+$ & $-$ & $-$ & $-$ \\
  \hline 5. Порождают ли две грамматики один и тот же язык ($L_{G1} =
  L_{G2}$)? & $+$ & $-$ & $-$ & $-$ \\
  \hline 6. Пусто ли \emph{пересечение} языков, порождаемых двумя
  грамматиками ($L_{G1} \cap L_{G2} = \varnothing$)? & $+$ & $-$ & $-$
  & $-$
  \\
  \hline 7. Для $\forall \alpha, \beta \in (T \cup N)^*$ выводимо ли
  $\alpha \stackrel[G]{}{\Rightarrow} \beta$ ($S
  \stackrel[G]{}{\Rightarrow} \gamma$; $\gamma \in T$)? & $+$ & $+$ &
  $+$ & $-$ \\
  \hline 8. Есть ли в языке, порождённом данной грамматикой, слово,
  выводимое более, чем один раз (т.\,е. является ли грамматика
  однозначной)? & $+$ & $-$ & $-$ & $-$ \\
  \hline 9. Существует ли однозначная грамматика того же языка,
  порождающая такой же язык? & $+$ & $-$ & $?$ & $+$ \\
  \hline
\end{longtable}

%%%%%%%%%%%%%%%%%%%%%%%%%%%%%%%%%%%%%%%%%%%%%%%%%%%%%%%%%%%%%%%%%%%%%%%%%%%%%%%%
%%%%%%%%%%%%%%%%%%%%%%%%%%%%%%%%%%%%%%%%%%%%%%%%%%%%%%%%%%%%%%%%%%%%%%%%%%%%%%%%
%%%%%%%%%%%%%%%%%%%%%%%%%%%%%%%%%%%%%%%%%%%%%%%%%%%%%%%%%%%%%%%%%%%%%%%%%%%%%%%%

\subsection{Задача анализа ТМИ как задача распознавания образов (ТС)}

\subsubsection{Вводные определения}

Существует \emph{большое многообразие} задач \textbf{\emph{анализа
    ТМИ}}, а также существует множество подходов к решению этих
задач. В наиболее общей форме задача анализа ТМИ может быть
сформулирована как задача \emph{распознавания образов} (теория
распознавания образов). Причём в качестве образа рассматривается
\emph{ТС}.

\begin{defin}
  \emph{ТС} ОУ (объекта анализа) будем называть совокупность
  изменяющихся в процессе производства испытаний, эксплуатации свойств
  (как есть) ОУ, характеризующих его функциональную пригодность в
  заданных условиях применения.
\end{defin}

ТС определяется путём оценивания \emph{параметров ТС}, множество
значений которых образует некоторое \emph{пространство параметров
  ТС}. Из этого следует, что определение (оценивание) ТС в процессе ИТО~---
заключается в:
\begin{enumerate}
\item[1)] \emph{указании некоторой точки};
\item[2)] отнесении её к \emph{определённой области} в
  \emph{пространстве параметров ТС}.
\end{enumerate}

Параметры ТС:
\begin{itemize}
\item измеряемые $X_и$;
\item вычисляемые $X_в$.
\end{itemize}

\begin{defin}
  \emph{Измеряемыми параметрами ТС} (ИПТС) являются представимые в
  виде значений ТМП показатели (характеристики) свойств ОУ.

  Совокупность измеряемых ПТС образует пространство ИПТС.

  \emph{Вычисляемыми параметрами ТС} (ВПТС) являются такие показатели
  (характеристики) свойств ОУ, которые могут быть вычислены по
  различным алгоритмам с использованием значений измеряемых параметров ТС.
\end{defin}

\begin{defin}
  \emph{Целью \textbf{анализа} ТМИ} как процесса является получение
  обобщённых оценок совокупности ПТС (с учётом конкретных целей
  применения ЛА на различных этапах его функционирования), значения которых в явном виде указывают:
  \begin{itemize}
  \item либо степень работоспособности ОУ;
  \item либо место и вид возникшей на борту неисправности;
  \item либо являются оценками прогнозируемых процессов и явлений с
    заданной точностью и интервалом прогноза;
  \item и т.\,п.
  \end{itemize}
\end{defin}

\emph{Получить цель анализа}~--- это значит указать точку в
пространстве ПТС, характеризующую ТС ОУ.

\emph{Цель анализа}~--- задаётся экспертом или лицом, принимающим
решение и осуществляющим управление ОУ.

Из этого следует, что анализ ТМИ~--- есть процесс получения оценок
параметров ТС, являющихся элементами цели анализа, вместе с оценками
показателей степени доверия полученным результатам.

Цель анализа может соответствовать точке как во всём пространстве ПТС,
так и в его подпространстве ИПТС. Разделяя ИТО на обработку и анализ,
мы тем самым считаем (содержательно) обработку одним из этапов
анализа.

В частности, весь анализ в отдельных случаях может заключаться только
в обработке ТМИ,~--- так будет тогда, когда целью анализа являются
элементы множества ИПТС.

Таким образом, \emph{задачей анализа ТМИ} является формирование и
отнесение точки в пространстве ТС к той или иной области, которой
сопоставляется, например, допустимое управляющее воздействие.

%%%%%%%%%%%%%%%%%%%%%%%%%%%%%%%%%%%%%%%%%%%%%%%%%%%%%%%%%%%%%%%%%%%%%%%%%%%%%%%%

\subsubsection{Формальная постановка задачи распознавания ТС}

Предварительно введём ряд базовых множеств.

$\Omega = \{\omega\}$~--- множество образов (ТС), подлежащих
распознаванию (классификации); $\Omega$~--- счётное множество.

$\Omega' = \{\omega\}$~--- обучающее множество образов (ТС);
$|\Omega'| = k$, $\Omega$ конечно.

$X = \{n\}$~--- множество параметров ТС; $X_и \cup X_в$.

$G = \{G_i\}$~--- множество классов образов (классов ТС).

\begin{figure}[h]
  \centering
  $$
  \xymatrix{
    \Omega'\ar[rr]^{\varphi'}\ar[ddrr]_(0.7){\eta} & & X_и\ar[d]^{\psi} \\
    & & X_в\ar[d]^{\delta} \\
    \Omega\ar[uu]_{\xi}\ar[rr]^{\zeta}\ar[uurr]^(0.7){\varphi}  & & G \\
  }
  $$
  \caption{Коммутативная диаграмма}
  \label{fig:cd}
\end{figure}

\begin{longtable}{p{0em}p{0em}c@{\;:\;}c@{\;$\to$\;}lp{26em}}
  $\mathrm{H}$ && $\eta$ & $\Omega'$ & $G$ & отношение обучения\\
  
  &\multirow{2}*{$\arraycolsep=0em
    \left\{\begin{array}{c}\\\\\end{array}\right.$} & $\varphi$ &
  $\Omega$ &
  \multirow{2}*{$\arraycolsep=0em
    \left.\begin{array}{l}X_и\\X_и\end{array}\right\}$} &
  \multirow{2}*{отношения наблюдения (измерения)} \\
  
  && $\varphi'$ & $\Omega'$ \\ 
  
  && $\psi$ & $X_и$ & $X_в$ & отношение вычисления вычислимости \\
    
  && $\delta$ & $X_в$ & $G$ & отношение интерпретации \\
  
  $\Xi$        && $\xi$ & $\Omega$ & $\Omega'$ & отношение обобщения \\
  $\mathrm{Z}$ &&  $\zeta$ & $\Omega$ & $G$ & отношение классификации
\end{longtable}

В теории формальных грамматик в качестве $\omega \in \Omega$
рассматривается слово (цепочка) $\alpha$ в некотором языке,
т.\,е. $\omega = \alpha$, $\Omega = L$, $\Omega' = L'$~--- обучающий
язык, $G_i$~--- грамматика, описывающая некоторый класс (множество)
слов $\alpha \subset L_i$.

\noindent\emph{Задача распознавания ТС}:

Дано: $\alpha \in L$.

Определить: принадлежность $\alpha \in \{L_1, L_2, \ldots\}$

\noindent Задача обучения системы распознавания

Дано: $\eta$~--- отношение обучения:

$\eta = \{ \langle\alpha,G_i\rangle \mid i \in I_G \}$

Определить: $G_i$, $\forall i \in I_G$

Как на этапе обучения, так и на этапе классификации процесс вычислений
связан с: 1) \emph{выбором} и 2) \emph{реализацией} каких-либо алгоритмов.
 

$A$~--- множество алгоритмов, позволяющих реализовать отношение
классификации $\zeta$ (или $\psi$): $A = \{a_i\}$.

\noindent $X_{a_i}^+$~--- множество входных операндов для $a \in A$;

\noindent $X_{a_i}^-$~--- множество выходных операндов для $a \in A$.

Тогда задача распознавания ТС разбивается на 2 подзадачи:
\begin{enumerate}
\item[1)] задача обучения;
\item[2)] задача непосредственно распознавания.
\end{enumerate}

\begin{enumerate}
\item Задача обучения.

  Дано: $X$, $\Omega'$, $c_i \in C'$, $G^i \stackrel[*]{}{\subset} G$,
  $\eta$~--- отношение обучения.
  \begin{enumerate}
  \item Построить: $A^i = \{a\}$.
  \item Выбрать $a^{op\,r} \in A^i$, $\tau_{a^{op\,r}} = \min\limits_{r \in
      I_a} \{\tau_{a^r}\}$~--- операторный анализ.
  \end{enumerate}

\item Задача классификации.

  Дано: $\omega \in \Omega$.

  Определить: $G^0 \colon d_{G^0} = \max\limits_{i\in{}I_G}
  \{d_i\}$~--- достоверный анализ.
\end{enumerate}

%%%%%%%%%%%%%%%%%%%%%%%%%%%%%%%%%%%%%%%%%%%%%%%%%%%%%%%%%%%%%%%%%%%%%%%%%%%%%%%%
%%%%%%%%%%%%%%%%%%%%%%%%%%%%%%%%%%%%%%%%%%%%%%%%%%%%%%%%%%%%%%%%%%%%%%%%%%%%%%%%
%%%%%%%%%%%%%%%%%%%%%%%%%%%%%%%%%%%%%%%%%%%%%%%%%%%%%%%%%%%%%%%%%%%%%%%%%%%%%%%%

\subsection{Автоматные грамматики и языки}

Как было указано ранее, самым простым по мощности и выразительной
способности является грамматика $G$ и порождаемый ею язык $L(G)$ типа
3~--- автоматный (регулярный) язык и грамматика.

\begin{rem}
Вспомним определение автоматной грамматики:

  ФГ~--- типа 3, если каждое её правило имеет вид: $A \to aB$, $A \to
  a$, где $A,B \in N$, $a \in T$.
\end{rem}

Мы говорили о праволинейной (правосторонней) ФГ $G$:
\begin{itemize}
\item что такое праволинейная ФГ?
\end{itemize}

Можно показать, что $L(G_{пр} \Leftrightarrow L(G_{лев})$, у которых
$G_{пр} \colon A \to aB \Leftrightarrow G_{лев} \colon A \to Ba$.

Доказать: $L(G_{пр}) \equiv L(G_{лев})$

Существует связь между КА и речевыми грамматиками. Это утверждают
следующие теории, которые к тому же указывают и путь получения \emph{КА по
автоматной грамматике} и \emph{автоматной грамматики по КА}:

\begin{theorem}
  Если $G = \langle T,N,P,S \rangle$~--- грамматика типа 3 (РГ), то
  существует КА \underline{без ...}: $S_{КА} = \langle
  A,Q,\delta,q_0,F \rangle$ такой, что допускаемое им множество
  входных слов совпадает в точности с множеством слов языка $L(G)$,
  причём:
  \begin{enumerate}
  \item $T \to A$;
  \item $N \to Q$;
  \item $S \to q_0$\quad $(S \in N, q_0 \in Q)$;
  \item $[B \to aD] \to [\delta(a,B) = D]$;
  \item $[B \to a] \to [\delta(a,B) \in F]$, $F$~--- множество
    финальных состояний КА $S_{КА}$;
  \end{enumerate}
  и наоборот.
\end{theorem}

\begin{theorem}
  Класс языков, допускаемых КА, в точности совпадает с классом
  регулярных языков.
\end{theorem}

\begin{ex}
  $G\colon P = \{S \to aB, B \to bS, B \to b\}$.
  \begin{figure}[h]
    \centering
    \begin{tabular}{c|cc}
      & a & b\\\hline
      S & B & $-$\\
      B & $-$ & S
    \end{tabular}
    \large\qquad$\xymatrix{S\ar@/^/[rr]|a & &
      \boxed{B}\ar@/^/[ll]|b}$
  \end{figure}
\end{ex}

%%%%%%%%%%%%%%%%%%%%%%%%%%%%%%%%%%%%%%%%%%%%%%%%%%%%%%%%%%%%%%%%%%%%%%%%%%%%%%%%

\subsubsection{Восстановление (синтез) ФГ типа 3}

В теории анализа ТМИ, как уже указывалось, рассматриваются 2 проблемы,
решение которых используется при распознавании ТС:
\begin{enumerate}
\item синтез ФГ (на этапе обучения);
\item анализ ФГ (ФЯ)~--- на этапе распознавания ТС.
\end{enumerate}

Рассмотрим 1-ую из них.

\begin{rem}
  В силу двух последних теорем, для синтеза ФГ типа 3 достаточно
  построить КА, допускающий заданный язык, а затем по этому КА
  построить ФГ.
\end{rem}

Путь синтеза ФГ типа 3.
$\boxed{L \to S \to S^0 \to G} \Leftrightarrow \boxed{L \to S \to S^0
  \to G}$

\begin{ex}
  $L = \{a^mb^nc^kd \mid m,n,k \in \mathbb{N}\}$
\end{ex}

%%%%%%%%%%%%%%%%%%%%%%%%%%%%%%%%%%%%%%%%%%%%%%%%%%%%%%%%%%%%%%%%%%%%%%%%%%%%%%%%

\subsubsection{Анализ ФГ типа 3}

Существует два подхода к анализу ФГ типа 3:
\begin{enumerate}
\item алгоритмы анализа, базирующиеся на алгоритмах анализа КА;
\item алгоритмы анализа, базирующиеся на непосредственно
  грамматическом выводе.
\end{enumerate}

\emph{1-й подход} заключается в следующем:
$$
\xymatrix{
  G\ar[r] & S_{КА}\ar[r] & \{\alpha \in L(G),\: \alpha \notin L(G)\}\\
  & \alpha \ar[u]
}$$

\emph{2-й подход}:
$$
\xymatrix{
  G\ar[r] & \{\alpha \in L(G),\: \alpha \notin L(G)\}\\
  \alpha \ar[u]
}$$

\begin{ex}
  \ 
  
  $\begin{aligned}[t]
    G\colon P &= \{S \to aS \mid aA,  \\
    \phantom{G\colon P } & \phantom{\mathrel{{} = \{}}
    \lefteqn{A}\phantom{S} \to bA \mid bS \mid b\}\\
    \alpha_1 &= aaabbbaab \\
    \alpha_2 &= aabbab
  \end{aligned}
  $

  Эта ФГ~--- недетерминированная, но не является неоднозначной.
\end{ex}

%%%%%%%%%%%%%%%%%%%%%%%%%%%%%%%%%%%%%%%%%%%%%%%%%%%%%%%%%%%%%%%%%%%%%%%%%%%%%%%%
%%%%%%%%%%%%%%%%%%%%%%%%%%%%%%%%%%%%%%%%%%%%%%%%%%%%%%%%%%%%%%%%%%%%%%%%%%%%%%%%
%%%%%%%%%%%%%%%%%%%%%%%%%%%%%%%%%%%%%%%%%%%%%%%%%%%%%%%%%%%%%%%%%%%%%%%%%%%%%%%%

\subsection{КС-грамматики и языки}

\noindent6. Рейуорд--Смит, В.\,Дж Теория формальных языков. Вводный курс /
Пер. с англ. А.\,Кузьмина, под ред. И.Г.\,Шестакова. М.: Радио и
связь, 1988~--- 128 с.

%%%%%%%%%%%%%%%%%%%%%%%%%%%%%%%%%%%%%%%%%%%%%%%%%%%%%%%%%%%%%%%%%%%%%%%%%%%%%%%%

\subsubsection{Нормальные формы КС грамматик}

Правила КС-грамматик имеют вид: 
$A \to \alpha$, где $A\in N,\;\alpha \in (T \cup N)^*$.

Такое ограничение является очень слабым. В силу этого КС-грамматики и
языки имеют очень большую порождающую мощность и поэтому большей
частью используются при распознавании ТС, в других областях прикладной
математики.

Однако это \emph{достоинство} (мощность КС-грамматик), определяемое
существенным ограничением $A \to \alpha$, является
\emph{недостатком} для изучения свойств таких КС грамматик. Вот почему
возникли различные \emph{нормальные формы} при задании $G$.

Нормальные формы КС-грамматик:
\begin{itemize}
\item НФ Хомского ($G_Х$)
\item НФ Грейбах ($G_Г$)
\end{itemize}

\begin{defin}
  КС-грамматика задана в нормальной форме Хомского, если её правила
  имеют вид:
  
  $$
  \begin{cases}
    A \to BC\\
    A \to a,&\text{где $A,B,C \in N$, $a \in T$.}
  \end{cases}
  $$
  
  Будем обозначать такую грамматику $G_Х$.
\end{defin}

\begin{defin}
  КС-грамматика задана в нормальной форме Грейбах, если её правила
  имеют вид:
  
  $$
    A \to a\alpha,\;\text{где $A \in N$, $a \in T$, $\alpha \in \{(N
      \cup T)^* \cup \varnothing\}$.}
  $$
  
  Будем обозначать такую грамматику $G_Г$.
\end{defin}

\begin{theorem}
  Для любой КС-грамматики общего вида всегда существует хотя бы одна
  $G_Х$, а также $G_Г$:
  $$
  \xymatrix{
    & G \ar[ld] \ar[rd]\\
    G_Х \ar@{<->}[rr]&&G_Г
  }
  $$
\end{theorem}

\begin{rem}
  Эта теорема говорит о том, что изучение свойств любой КС-грамматики
  $G$ можно производить на эквивалентной ей $G_Х$ или $G_Г$, которые
  порождают тот же язык $L(G)$.
\end{rem}

Алгоритмы эквивалентных преобразований КС-грамматик приведены в [6]. и
желающие (в силу недостатка нашего времени) могут изучить их сами.

%%%%%%%%%%%%%%%%%%%%%%%%%%%%%%%%%%%%%%%%%%%%%%%%%%%%%%%%%%%%%%%%%%%%%%%%%%%%%%%%

\subsubsection{Анализ КС-грамматик}

Использование рассмотренных нормальных форм позволяет строить более
простые алгоритмы анализа КС-языков и грамматик.

Однако, как и в случае анализа автоматных грамматик, анализ
КС-грамматик может быть осуществлён двумя путями:
\begin{enumerate}
\item 
  $
  \xymatrix{
    G \ar[rr] & &
    {\left\{\begin{aligned}
      \alpha &\in L(G),\\
      \alpha &\notin L(G)
    \end{aligned}\right\}}\\ 
    \alpha \ar[u]
    }
  $
\item 
  $
  \xymatrix{
    G \ar[r] & G_{Х,Г} \ar[r] & S_{МП} \ar[r] & 
    {\left\{\begin{aligned}
      \alpha &\in L(G),\\
      \alpha &\notin L(G)
    \end{aligned}\right\}}\\ 
    &&\alpha \ar[u]
    }
  $
\end{enumerate}

1-й путь реализуется использованием большого количества алгоритмов:
\begin{itemize}
\item алгоритм на основе матрицы предшествования;
\item алгоритм Кнута;
\item алгоритм с использованием степенных рядов;
\item алгоритм Эрли;
\item \dots
\end{itemize}

2-й путь~--- с использованием $S_{МП}$.

%%%%%%%%%%%%%%%%%%%%%%%%%%%%%%%%%%%%%%%%%%%%%%%%%%%%%%%%%%%%%%%%%%%%%%%%%%%%%%%%

\subsubsection{Понятие об автомате с магазинной памятью}

МП-автомат (автомат с магазинной памятью является дальнейшим развитием~КА:
$$
\xymatrix{
  \begin{tabular}{c}
    $
    \overbrace{
      \begin{tabular}{|c|c|c|c|c|c|c|c|}
        \hline
        $x_1$ & $x_2$ & $x_3$ & \dots & $x_i$ & \dots & $x_{n-1}$ & $x_n$\\
        \hline
      \end{tabular}
    }^{\displaystyle\alpha}
    $
  \end{tabular}
  \\
  \fbox{\parbox[.7\height]{7em}{%
      \begin{center}
        Устройство управления МП-автоматом
      \end{center}
    }%
  } \ar@<2em>@{=>}[r] \ar@{=>}[u]
  &
  {\begin{tabular}[t]{|c|}
      \hline
      $z_i$\\\hline
      $z_{i-1}$\\\hline
      \dots\\\hline
      $z_0$\\\hline
    \end{tabular}}
  & \text{\parbox{6em}{
      \begin{center}
        стек (магазинная память)
      \end{center}
    }}
}
$$

\begin{defin}
  \emph{МП-автомат} определяется как \emph{семёрка}:
  \begin{equation*}
    S_{МП} = \langle X,Q,V,\delta,q_0,Q_F \rangle\,,
  \end{equation*}
\noindent где $X$~--- конечный алфавит входных символов МПА (как в КА);

$Q$~--- конечный алфавит внутренних состояний МПА (как в КА);

$q_0 \in Q$~--- начальное состояние;

$V$~--- конечное множество магазинных символов;

$z_0$~--- начальный символ магазина, находящийся всегда на дне;

$Q_F \subseteq Q$~--- множество заключительных (финальных) состояний
(как в КА);

\hangindent=4em$\sigma$~--- функция переходов:
\begin{equation*}
  \delta \colon Q \times X \times V \to \mathcal{B} (Q \times V^*)\,,
\end{equation*}
которая отображает тройки, образованные \emph{состоянием, входным
  символом, магазинным символом (доступным)~--- вершиной магазина}, в
множество всех подмножеств пар вида \emph{внутреннее состояние слова в
алфавите магазинных символов}.\\
$\mathcal{B}$~--- операция, называемая \emph{булеаном}.\\
\end{defin}

МП-автомат функционирует сменяя свои конфигурации.

\begin{defin}
  \emph{Конфигурация} МП-автомата~--- это элемент множества пар
  \begin{equation*}
    Q \times V^*=\{(q_i,\gamma)\}\,,
  \end{equation*}
  где $q_i \in Q_i$~--- текущее состояние МП-автомата;

  $\gamma \in V^*$ (слово в стеке магазинных символов)~--- текущее
  содержимое стека
\end{defin}

Смена конфигураций называется \emph{движением МП-автомата}.

МП-автомат используется как устройство, \emph{распознающее}
(допускающее или отвергающее) цепочки (стека), записанные на входной
ленте, просматривая символы этой цепочки слева направо \emph{без
  возвратов}.

\noindent Если:
\begin{enumerate}
\item автомат находится в конфигурации $(q,\gamma v)$ ($v$~--- верхний
  доступный для обозрения символ в стеке;
\item считывающая головка обозревает ячейку входной ленты с символом $a_i$\,;
\item $(q',\gamma\beta) \in \delta(q,x_i,v)$\,,
\end{enumerate}
то МП-автомат может изменить конфигурацию на $(q',\gamma')$, а символ
$x_i$ считается допущенным.

Для обозначения такого перехода используется запись:
\begin{equation*}
  x_i\colon (q,\gamma v) \to (q', \gamma\beta)
\end{equation*}

Если $x_i = \varepsilon$, то такой переход называется
\emph{$\varepsilon$-переходом}, т.\,е. автомат переходит из одной
конфигурации в другую \emph{без ввода входного символа}: входное слово
$\alpha = {x_i}_1,{x_i}_2,\ldots,{x_i}_n$ \emph{допускается
  МП-автоматом}, если, начиная с первого символа ${x_i}_1$ и начальной
конфигурации, автомат переходит после введения последнего символа
${x_i}_n$ в конфигурацию $(q_i,v_0\gamma)$, у которой $q_i \in Q_F$.

\begin{ex}
  Рассмотрим МП-автомат, допускающий цепочки языка $L = \{a^nb^n \mid
  n>1\}$.

  $L$~--- язык Дика: в программировании~--- язык описания скобочных
  структур.

  $L$~--- не может быть описан КА и автоматной грамматикой.
  
  $$S_{МП} = \langle X,Q,V,\delta,q_0,v_0,Q_F \rangle\,,$$
  где $X = \{a,b\}$, $Q = \{q_0,q_1,q_2,q_3\}$, $V = \{v\}$, $Q_F = \{q_3\}$;
  \begin{tabbing}
    $\delta(q_0,a,v_0) = \{(q_1,v)\}$\qquad\=$\delta(q_1,a,v) =
    \{(q_1,vv)\}$\=\qquad\=$\delta(q_1,b,vv) = \{(q_2,v)\}$\\
    $\delta(q_2,b,vv) = \{(q_2,v)\}$ \>
    $\delta(q_2,b,v_0v)=\{(q_3,v_0)\}$
  \end{tabbing}
  т.\,к. $\delta\colon Q \times X \times V \to B(Q \times V^*)\,.$

  $\alpha = a\ a\ a\ b\ b\ b$

  \begin{table}[h]
    \centering
    \begin{tabular}{c @{\qquad} c @{\qquad} c @{\qquad} l r}
    \hline\hline
    \textbf{№ шага} & $\mathbf{q}$ & $\mathbf{x_i}$ & \textbf{Стек} &
    \textbf{Оставшаяся часть слова} $\boldsymbol{\alpha}$ \\\hline
    0 & $q_0$ & $a$ & $v_0$ & $a\ a\ a\ b\ b\ b$\phantom{\qquad\qquad}\\
    1 & $q_1$ & $a$ & $v_0v$ & $a\ a\ b\ b\ b$\phantom{\qquad\qquad}\\
    2 & $q_1$ & $a$ & $v_0vv$ & $a\ b\ b\ b$\phantom{\qquad\qquad}\\
    3 & $q_1$ & $b$ & $v_0vvv$ & $b\ b\ b$\phantom{\qquad\qquad}\\
    4 & $q_2$ & $b$ & $v_0vv$ & $b\ b$\phantom{\qquad\qquad}\\
    5 & $q_2$ & $b$ & $v_0v$ & $b$\phantom{\qquad\qquad}\\
    6 & $q_3$ & $-$ & $v_0$\\\hline\hline
  \end{tabular}
\end{table}

$G\colon P=\{S \to aSb; S \to ab\}$
\end{ex}

\begin{theorem}
  Для любой КС грамматики $G$ существует такой МП автомат $S_{МП}$ и
  наоборот, что
  $$L(G) = L(S_{МП})\,.$$
\end{theorem}

%%%%%%%%%%%%%%%%%%%%%%%%%%%%%%%%%%%%%%%%%%%%%%%%%%%%%%%%%%%%%%%%%%%%%%%%%%%%%%%%

\subsubsection{Анализ КС-языков с использованием систем уравнений}

Как это отмечено ранее, существует большое многообразие алгоритмов
анализа КС-языков. Одним из них является алгоритм, основанный на
анализе степенных рядов.

\begin{defin}
  Существует $T = \{x_j \mid j \in I_T\}$~--- территориальный словарь
  некоторого языка $L(G)$, порождаемого КС-грамматикой $G$.

  Введем отображение $r$:
  $$\begin{cases}
    r \colon \alpha \to z & \text{или}\\
    r \colon T^* \to Z\,,
  \end{cases}$$
  которое каждому $\alpha \in T^*$ ставит в соответствие некоторый
  элемент $z \in Z$ (числовое множество).
\end{defin}

Тогда совокупность пар
$$
R = \sum_{\alpha \in L(G)} \langle z,\alpha \rangle
$$
назовём \emph{степенным рядом} языка $L(G)$ для КС-грамматики $G$, а
элементы $z \in Z$~--- \emph{коэффициентами} (или степенью)
\emph{соответствующего слова $\alpha$}.

В зависимости от состава множества $Z$ различают следующие степенные
ряды:
\begin{list}{\arabic{N})}{\usecounter{N}}
\item $Z = \{0,1\}$ $\Rightarrow$ $R$~--- называется
  характеристическим рядом;
\item $Z = \{0,1,2,3,\ldots\}$ $\Rightarrow$ $R$~--- ряд с целыми
  положительными коэффициентами, $z \in Z$~--- кратность слов;
\item $Z = \{[0,1]\}$ $\Rightarrow$ $R$~--- ряд с нормированными
  коэффициентами, $z$~--- вероятность слова $\alpha$.
\end{list}

Рассмотрим $Z = \{0,1\}$~--- характеристический степенной ряд, причём
при определении такого степенного ряда будем указывать только те члены
ряда, у которых $Z = 1$ ($Z \ne 0$).

Определим операции сложения и умножения на множестве степенных рядов.

\begin{defin}
  \emph{Сложением} двух рядов $R_1$ и $R_2$ назовём бинарную операцию,
  в результате выполнения которой формируется ряд $R$:
  $$R = R_1 + R_2$$
  такой, что коэффициенты $z$ в $R$ при каждой цепочке $\alpha$ есть
  сумма коэффициентов при аналогичных цепочках в рядах $R_1$ и $R_2$,
  причём:
  $$
  \begin{cases}
    \alpha + \alpha = 1\cdot\alpha + 1\cdot\alpha = (1 + 1)\cdot\alpha =
    1\cdot\alpha = \alpha\,;\\
    \alpha + 0 = 1\cdot\alpha + 0\cdot\alpha = (1 + 0)\cdot\alpha =
    1\cdot\alpha = \alpha\,.\\
  \end{cases}
  $$
  т.\,е. $z\alpha = z_1\alpha + z_2\alpha$, где $z = \max (z_1 +
  z_2)$, $\alpha \in T^*$.
\end{defin}

\begin{defin}
  \emph{Произведением} двух рядов $R_1$ и $R_2$ называют бинарную
  операцию, выполнение которой приводит к ряду $R$:
  $$R = R_1 \cdot R_2\,,$$
  такому, что коэффициенты $z$ формируются следующим образом:
  $$z = z_{i_1} z_{j_1} + z_{i_2} z_{j_2} + \ldots\,,$$
  для которых верно условие:
  $$\alpha_{i_1}\cdot\alpha_{j_1} = \alpha_{i_2}\cdot\alpha_{j_2} =
  \ldots = \alpha\,.$$
  где $\alpha_i\cdot\alpha_j$~--- операция конкатенации цепочек
  $\alpha_i$ и $\alpha_j$.
  
  Например, $ab \cdot bc = abbc$.
\end{defin}

Из рассмотренных определений вытекает следующая теорема:
\begin{theorem}
  Каждой КС-грамматике $G$ соответствует степенной ряд $R$.
\end{theorem}

Пусть $S, A, B \in N$~--- нетерминалы в $G$; $S$~--- аксиома.

Сопоставим каждому элементу из $N$ языковую переменную:
$$\begin{tabular}{c@{\;---\;}c}
  $S$ & $\gamma_1$\\
  $A$ & $\gamma_2$\\
  $B$ & $\gamma_3$\\
\end{tabular}$$

\begin{defin}
  Языковой переменной $\gamma$ назовём такую переменную, которая
  принимает значения на множестве $T^*$, т.\,е. $\gamma$~--- некоторый
  язык в алфавите $T$.
\end{defin}

Распределим правила $R$ грамматики $G$ на следующие группы:
\begin{list}{\arabic{N})}{\usecounter{N}}
\item $S \to \varphi_1$, $S \to \varphi_2$, \ldots, $S \to \varphi_k$\,,
\item $A \to \psi_1$, $A \to \psi_2$, \ldots, $A \to \psi_l$\,,
\item $B \to \theta_1$, $B \to \theta_2$, \ldots, $B \to \theta_m$\,.
\end{list}
где $\varphi_i$, $\psi_i$, $\theta_i$~--- слова в алфавите $(T \cup
N)^*$ (правые части правил).

Заменим в цепочках $\varphi_i$, $\psi_i$, $\theta_i$ каждое вхождение
нетерминального символа соответствующей переменной, а замет объединим\\
все цепочки $\varphi$ в формальное выражение $f$:
$$f = \varphi_1 + \varphi_2 + \ldots + \varphi_k\,,$$
все цепочки $\varphi$ в формальное выражение $g$:
$$g = \psi_1 + \psi_2 + \ldots + \psi_l\,,$$
все цепочки $\varphi$ в формальное выражение $h$:
$$f = \theta_1 + \theta_2 + \ldots + \theta_m\,,$$
и т.\,д.

Тогда получим систему уравнений:
$$
\begin{cases}
  \gamma_1 = f(\gamma_1, \gamma_2, \gamma_3, \ldots)\,;\\
  \gamma_2 = g(\gamma_1, \gamma_2, \gamma_3, \ldots)\,;\\
  \gamma_3 = h(\gamma_1, \gamma_2, \gamma_3, \ldots)\,;\\
  \ldots
\end{cases}
$$

Решение полученной системы уравнений с целью определения значений
языковых переменных $\gamma_1, \gamma_2, \gamma_3, \ldots$
осуществляется методом последовательных приближений.

На начальном (нулевом) шаге языковым переменным присваиваются значения
$0$.
$$
\begin{cases}
  \gamma_1^{(0)} = 0\,,\\
  \gamma_2^{(0)} = 0\,,\\
  \gamma_3^{(0)} = 0\,,\\
  \ldots\\
\end{cases}
$$

На последующих шагах в правые части уравнений вместо языковых
переменных $\gamma_1, \gamma_2, \gamma_3, \ldots$, входящих в состав
\emph{степенных рядов} $f, g, h, \ldots$, подставляются их значения с
предыдущего шага, т.\,е. формально на каждом $k$-ом шаге решение
уравнений имеет вид:
$$
\begin{cases}
  \gamma_1^{(k)} = f(\gamma_1^{(k-1)}, \gamma_2^{(k-1)},
  \gamma_3^{(k-1)}, \ldots)\,;\\
  \gamma_2^{(k)} = g(\gamma_1^{(k-1)}, \gamma_2^{(k-1)},
  \gamma_3^{(k-1)}, \ldots)\,;\\ 
  \gamma_3^{(k)} = h(\gamma_1^{(k-1)}, \gamma_2^{(k-1)},
  \gamma_3^{(k-1)}, \ldots)\,;\\ 
  \ldots
\end{cases}
$$

В общем случае, исходя из начального приближения, будем
последовательно получать значения языковых переменных, а в качестве
индуктивного сформированного предельного решения будем иметь
совокупность таких значений языковых переменных:
$$
\begin{cases}
  \gamma_1 = \gamma_1^{(1)} + \gamma_1^{(2)} + \gamma_1^{(3)} + \ldots
  + \gamma_1^{(k)} + \ldots ;\\
  \gamma_2 = \gamma_2^{(1)} + \gamma_2^{(2)} + \gamma_2^{(3)} + \ldots
  + \gamma_2^{(k)} + \ldots ;\\
  \gamma_3 = \gamma_3^{(1)} + \gamma_3^{(2)} + \gamma_3^{(3)} + \ldots
  + \gamma_3^{(k)} + \ldots ;\\
  \ldots\\
\end{cases}
$$

Отметим, что значение каждой языковой переменной, представляющей
степенной ряд, образует, соответственно, язык:
$$
\begin{cases}
  \gamma_1 = L_S\,,\\
  \gamma_2 = L_A\,,\\
  \gamma_3 = L_B\,,\\
  \ldots\\
\end{cases}
$$
Язык при аксиоме $L_S$ есть $L_S = L(G)$.

Поскольку результатом анализа цепочки $\alpha$ по заданной грамматике
$G$ является вывод о принадлежности $\alpha$ языку $L(G)$,
порождаемому грамматикой $G$, то процесс анализа реализуется следующим
образом.

Мы имеем входное слово $\alpha$ длины $n$ на каждом $i$-ом шаге
приближения среди пар $\langle z_j, \alpha_j \rangle$ из состава
степенного ряда при аксиоме $S$ ищется та пара, у которой
$$
\begin{cases}
  z_j = 1\,,\\
  \alpha_j = \alpha\,.\\
\end{cases}
$$

Если такой пары нет, то ищется решение на следующем шаге приближения и
т.\,д.

Такие шаги делаются до тех пор, пока:
\begin{itemize}
\item не будет найден элемент $\langle z_j, \alpha_j \rangle$,
  $\alpha_j = \alpha$ в значении $\gamma_1$ (при аксиоме) ($\alpha \in L(G)$);
\item не будет произведено такое максимально необходимое
  (гарантированное) число шагов $R_{max}$, что все возможные цепочки
  $\alpha_j$ длины $n$ уже порождены, но ни одна из них не совпадает с
  $\alpha$ ($\alpha \in L(G)$).
\end{itemize}

\begin{ex}
  Дана грамматика $G$:
  \begin{center}
    \begin{tabular}{l}
      $S \to aB \mid bA$,\\
      $A \to aS \mid bAA \mid a$,\\
      $B \to bS \mid aBB \mid b$.\\
    \end{tabular}
  \end{center}
  $L(G)$~--- язык, включающий все те и только те слова, которые
  состоят из равного числа символов $a$ и $b$ (например, $\alpha =
  bbaa$, $\alpha = abba$).
\end{ex}


%%% Local Variables: 
%%% mode: latex
%%% TeX-master: "lections"
%%% End: 

\section{Структурно-стохастические модели (ССМ) функционирования БС}

Основная задача анализа ТМИ~--- получение значений параметров
\emph{цели анализа}.

При сборе ТМИ вследствие разного рода неучтённых и непредсказуемых
факторов, разрушающих информацию, возможны различные по природе
искажения значений параметров ТС $X_и$ (измеряемых). Это, в свою
очередь, может повлиять на определение значений вычисляемых параметров
$X_в$, и, как следствие, к неправильному оцениванию параметров цели
анализа $C$.

Поэтому в системах анализа ТМИ необходимо использовать модели
стохастического типа.

При использовании структурных (лингвистических) моделей
функционирования БС такими моделями являются \emph{стохастические
  грамматики (СГ), стохастические автоматы (СА), стохастические языки
  (СЯ)}, являющиеся \emph{структурно-стохастическими моделями
  функционирования БС}.

Перейдем к их рассмотрению.

%%%%%%%%%%%%%%%%%%%%%%%%%%%%%%%%%%%%%%%%%%%%%%%%%%%%%%%%%%%%%%%%%%%%%%%%%%%%%%%%
%%%%%%%%%%%%%%%%%%%%%%%%%%%%%%%%%%%%%%%%%%%%%%%%%%%%%%%%%%%%%%%%%%%%%%%%%%%%%%%%

\subsection{Определение и основные типы структурно-стохастических моделей}

\subsubsection{Определение и классификация СГ}

Основным понятием при рассмотрении ССМ является понятие СГ. Дадим её
определение.

\begin{defin}
  Стохастическая грамматика (СГ) $G_s$~--- есть кортеж (восьмёрка):
  \begin{equation}
    \label{eq:Gs_general}
    G_s = \langle N, T, R, S, \Pi_S, \pi_S, \Pi_R, \pi_R \rangle\,,
  \end{equation}
  где\begin{tabular}[t]{r@{\;---\;}p{.88\textwidth}@{}}
    $N$ & конечное множество нетерминальных символов (алфавит
    нетерминалов), вспомогательных символов;\\
    
    $T$ & конечное множество терминальных символов (алфавит
    терминалов);\\
    
    $R$ & конечное множество правил вывода;\\
    
    $S$ & конечное множество аксиом (в отличие от нестохастических
    структурных моделей $|S| \geqslant 1$);\\
    
    $\Pi_S$ & вероятностная мера $p \in \Pi_S \in [0,1]$;\\
    
    $\pi_S$ & отображение, сопоставляющее каждому элементу $A_i \in S$
    его вероятностную меру $p \in \Pi_S$: $\pi_S \colon S \to \Pi_S$;\\
    
    $\Pi_R$ & вероятностная мера $p \in \Pi_R \in [0,1]$ (аналогично
    $\Pi_S$);\\
    
    $\pi_R$ & отображение, сопоставляющее каждому правилу подстановки
    $r \in R$ его вероятностную меру $p \in \Pi_R$, $\pi_R \colon R \to
    \Pi_R$.\\
  \end{tabular}
\end{defin}

Такое определение СГ является наиболее общим.

\begin{rem}
  Обычно рассматривают СГ с $|S| = 1$. В этом случае получается СГ:
  \begin{equation}
    \label{eq:Gs}
    G_S = \langle S, N, T, R, \Pi_R, \pi_R \rangle\,,
  \end{equation}
  т.\,к. $\pi_S \colon \langle A_i, 1 \rangle$, где $A_i$~--- аксиома.
\end{rem}

Как видно из определения, в состав каждой СГ входит известная нам
четвёрка $\langle N, T, R, S \rangle$, определённая ранее как
формальная грамматика. Такая ФГ получила специальное название:

\begin{defin}
  Формальная грамматика, лежащая в основе СГ $G_S$ \eqref{eq:Gs},
  т.\,е. СГ без вероятностей правил перехода, называется
  \emph{характеристической грамматикой} $G_S^0$ для данной СГ $G_S$:
  \begin{equation}
    \label{eq:Gs0}
    G_S^0 = \langle N, T, R, S \rangle\,.
  \end{equation}
\end{defin}

\begin{rem}
  Поскольку для каждой СГ существует её характеристическая, то все СГ
  можно классифицировать по типам:
  \begin{enumerate}
  \item СГ типа 0;
  \item СГ типа 1;
  \item СГ типа 2;
  \item СГ типа 3.
  \end{enumerate}
\end{rem}

\begin{ex}
  $G_S = \langle T, N, S, R, \pi_R \rangle$:
  \begin{center}
    \begin{tabular}{l}
      $T = \{a\}$,\\
      $N = \{S\}$,\\
      $S$,\\
      $R = \{S \xrightarrow{0{,}8} aSa,\; S \xrightarrow{0{,}2} aa\}$.\\
    \end{tabular}
  \end{center}
\end{ex}

\emph{Обозначение}. Правила СГ будем обозначать (для СГ типа 2 и 3):
\begin{equation}
  \label{eq:rj}
  r_j \colon A_j \xrightarrow{P_j} \beta_k \text{ или } A_j
  \xrightarrow{P_{i_k}} \beta_k\,,
\end{equation}
где\begin{tabular}[t]{r@{\;---\;}l@{}}
  $A_i \in N$ & левая часть правила;\\
  $\beta_k \in (T \cup N)^*$ & правая часть правил;\\
  $P_j$ или $P_{i_k}$ & вероятность применения правила $r_j \colon P_j
  = P_{i_k} = P(r_j)$.
\end{tabular}

Так же, как каждая ФГ $G$ порождает некоторый язык $L(G)$, так и
каждая СГ $G_S$ порождает стохастический язык $L(G_S) = L_S$.

Характерной особенностью СЯ является то, что каждое слово $\alpha \in
L_S$ СЯ имеет свою вероятностную меру $P(\alpha)$. Это равносильно
тому, что терминальный вывод цепочек $\alpha$ из аксиомы грамматики,
$S$ имеет вероятность:
$$P(S \stackrel[G_S]{*}{\Rightarrow} \alpha) = P(\alpha)\,.$$
Причём верно, что
$$
\sum_{\alpha \in L_S} P(\alpha) = 1\,,
$$
т.\,е. сумма вероятностей всех слов СЯ $L_S$ равна единице.

Это означает, что если при анализе ТМИ каждое событие предоставляется
некоторым словом $\alpha \in L_S$, то весь СЯ $L(G_S)$ образует
\emph{полную группу событий}.

\begin{defin}
  СГ $G_S$, порождающая СЯ:
  $$L(G_S) = \{ \langle \alpha, P(\alpha) \rangle \mid \alpha \in T^k,
  P(\alpha) = P(S \stackrel[G_S^0]{*}{\Rightarrow} \alpha) \}$$
  такой, что
  $$
  \sum_{\alpha \in L_S} P(\alpha) = 1\,,
  $$
  называется \emph{согласованной СГ}
\end{defin}

\begin{ex}
  $G_S \cdot R_S = \{ S \xrightarrow{0{,}8} aSa, S \xrightarrow{0{,}2}
  aa \}$
  
  $\alpha = aaaaaa$; $P(\alpha)$~--- ?
  
  \begin{figure}[h]
    \Tree [.S a [.S a [.S a a ] a ] a ]
    \caption{Дерево вывода}
  \end{figure}

  $S \stackrel{0{,}8}{\Rightarrow} aSa \xrightarrow{0{,}8}{\Rightarrow}
  aaSaa \stackrel{0{,}2}{\Rightarrow} aaaaaa$

  $P(\alpha) = P(aaaaaa) = 0{,}8 \cdot 0{,}8 \cdot 0{,}2 = 0{,}128$.

Определим согласованность СГ $G_S$:
\begin{equation*}
  \begin{split}
    \sum_{\alpha \in L(G_S)} P(\alpha) &= 0{,}2 + 0{,}2 \cdot 0{,}8 +
    0{,}2 \cdot 0{,}8 \cdot 0{,}8 + \ldots = \sum_{n=0}^{\infty} 0{,}2
    \cdot 0{,}8^n = \\
    &= 0{,}2 \cdot \sum_{n=0}^{\infty}0{,}8 = 0{,}2 \cdot
    \frac{1}{1-0{,}8} = 0{,}2 \frac{1}{0{,}2} = 1.
  \end{split}
\end{equation*}
\end{ex}

%%%%%%%%%%%%%%%%%%%%%%%%%%%%%%%%%%%%%%%%%%%%%%%%%%%%%%%%%%%%%%%%%%%%%%%%%%%%%%%%

\subsubsection{Стохастические грамматики и вероятностные автоматы}

Формальные грамматики типа 3, как известно, эквивалентны по своей
мощности КА: т.\,е. языки, порождаемые ФГ типа 3 и КА~--- эквиваленты.

Аналогичная связь существует между СГ типа 3 и вероятностными
(стохастическими) конечными автоматами (\emph{СА}). Рассмотрим эту
аналогию.

Вспомним определением СА (вероятностного):
\begin{defin}
  СА называется кортеж (пятёрка)
  $$
  S_S = \langle X, Q, \pi_0, \delta_S, Q_F \rangle\,.
  $$
  
  \begin{tabular}[t]{r@{\;---\;}p{.85\textwidth}@{}}
    $X$ & входной алфавит СА;\\
    $Q$ & алфавит внутренних состояний СА;\\
    $\pi_0$ & вектор-строка (начальное распределение состояний;\\
    $Q_F$ & множество финальных состояний, $Q_F \subseteq Q$;\\
    $\delta_S$ & стохастическая (вероятностная) функция переходов.\\
  \end{tabular}
\end{defin}

\begin{ex}
  \label{ex:KA}  
  Для сравнения приведём вначале нестохастический КА.
\begin{itemize}
\item \emph{нестохастический КА} (задан графом переходов) 
  $$
  \boxed{\boxed{\delta \colon X \times Q \to Q}}
  $$
  \begin{figure}[h]
    \centering
    \begin{equation*}
      \entrymodifiers={+++[o][F-]}
      \SelectTips{cm}{}
      \xymatrix @+1pc {
        1 \ar[r]^a 
        & 2 \ar[r]^b \ar@/_25pt/[rr]^b \ar`ur^l/12pt[]`^dr[] _a []
        & 3 \ar[r]^b \ar`ur^l[]`^dr[] _b [] % \ar@(ru,lu)[]_b
        & *+++[o][F=]{\#}
      }
    \end{equation*}
    
    $$
    M =
    \begin{tabular}{c | c c c c}
     & 1 & 2 & 3 & $\#$\\\hline
     1 & & $a$\\
     2 & & $a$ & $b$ & $b$\\
     3 & & & $b$ & $b$\\
     $\#$\\
    \end{tabular}
    \qquad L(S) = \{a^n b^m \mid n,m \geqslant 1\}
    $$
    \caption{Нестохастический конечный автомат}
    \label{fig:KA}
  \end{figure}
\item стохастический КА (задан 3-хмерной автоматной матрицей) 
  $$
  \boxed{\boxed{\delta_S \colon X \times Q \times Q \to P}}
  $$

  $$
    M = \;\begin{tabular}{@{}c@{}}
      \begin{picture}(160,120)(-15,-15)
        \multiput(0,0)(20,0){5}%
        {\line(0,1){80}}
        \multiput(0,0)(0,20){5}%
        {\line(1,0){80}}
        
        \multiput(0,80)(12,4){6}%
        {\line(1,0){80}}
        \multiput(0,80)(20,0){5}%
        {\line(3,1){60}}
        
        \multiput(80,0)(0,20){5}%
        {\line(3,1){60}}
        \multiput(80,0)(12,4){6}%
        {\line(0,1){80}}
        
        \put(-15,68){$q_1$}
        \put(-15,48){$q_2$}
        \put(-15,28){$q_3$}
        \put(-10,5){$\vdots$}
        
        \put(7,-10){$q_1$}
        \put(27,-10){$q_2$}
        \put(47,-10){$q_3$}
        \put(63,-10){$\cdots$}

        \put(84,-8){$x_1$}
        \put(96,-4){$x_2$}
        \put(108,0){$x_3$}
        \multiput(123,6)(6,2){3}%
        {.}
      \end{picture}
    \end{tabular}
    $$

    $$
    M = \;\begin{tabular}{@{}c@{}}
      \begin{picture}(120,105)(-15,-15)
        \multiput(0,0)(20,0){5}%
        {\line(0,1){80}}
        \multiput(0,0)(0,20){5}%
        {\line(1,0){80}}
        
        \multiput(0,80)(12,4){3}%
        {\line(1,0){80}}
        \multiput(0,80)(20,0){5}%
        {\line(3,1){24}}

        \multiput(80,0)(0,20){5}%
        {\line(3,1){24}}
        \multiput(80,0)(12,4){3}%
        {\line(0,1){80}}
        
        \put(-12,66){$1$}
        \put(-12,46){$2$}
        \put(-12,26){$3$}
        \put(-13,6){$\#$}

        \put(7,-12){$1$}
        \put(27,-12){$2$}
        \put(47,-12){$3$}
        \put(65,-12){$\#$}

        \put(84,-8){$a$}
        \put(96,-4){$b$}
      \end{picture}
    \end{tabular}
    $$
  \begin{figure}[h]
    \centering
    $$
    M(a) = \begin{tabular}{c | c c c c}
      & 1 & 2 & 3 & $\#$\\\hline
      1 & & 1\\
      2 & & 0{,}4\\
      3 \\
      $\#$\\
    \end{tabular}
    \qquad M(b) = \begin{tabular}{c | c c c c}
      & 1 & 2 & 3 & $\#$\\\hline
      1\\
      2 & & & 0{,}5 & 0{,}1\\
      3 & & & 0{,}6 & 0{,}4\\
      $\#$\\
    \end{tabular}
    $$
    \caption{Стохастический конечный автомат}
    \label{fig:KAs}
  \end{figure}
  \begin{rem}
    Сумма элементов автоматной матрицы $M$ (3-хмерной) по слоям должна
    быть равна $1$.
  \end{rem}
\end{itemize}
\end{ex}

Соответствие между языком, допускаемым СА, и марковским процессом
устанавливается следующей теоремой.
\begin{theorem}
  Любой однородный марковский процесс с дискретным временем и конечным
  множеством состояний может быть представлен как язык, допускаемый
  \emph{стохастическим конечным автоматом}, т.\,е. если известен
  $X(t)$~--- марковский процесс, то от него можно перейти к СА $S_S$
  такому, что $L(S_S) = X(t)$.
\end{theorem}

Имеет место также следующая теорема, устанавливающая связь между СА и
СГ типа 3.

\begin{theorem}
  Для любого стохастического конечного автомата (СА) $S_S$ существует
  такая СГ типа 3, $G_S$, что:
  $$L(S_S) = L(G_S)$$
  и наоборот.
\end{theorem}

\emph{Алгоритм постоения $G_S$ типа 3 по СА $S_S$}:

$$S_S = \langle X, Q, \delta_S, Q_F \rangle\,,\qquad G_S = \langle T,
N, S, R \rangle\,,$$ тогда:
\begin{list}{\arabic{N})}{\usecounter{N}}
\item $T = X$;
\item $N = Q \mathbin{\backslash} \{\#\} = Q \mathbin{\backslash}
  Q_F$;
\item $[A_i \xrightarrow{P_{ijk}} x_j A_k]
  \Leftrightarrow [\delta(x_j,q_i,q_k) = p_{ijk}]$,\\
  $[A_i \xrightarrow{P_{ij}} x_j] \Leftrightarrow [\delta(x_j,q_i,\#)
  = p_{ij}]$;
\item $S = A_i = q_i$,\quad $S,A_i \in N$,\quad $q_i \in Q$,\quad
  $\pi_0 = \langle \pi_{0,1},\pi_{0,2},\ldots,\pi_{0,i},\ldots
  \rangle$,\quad $\pi_{0,i} = 1$.
\end{list}

\begin{ex}
  Построить $G_S$ по $S_S$ (см. пример на стр. \pageref{ex:KA})
  
  $G_S = \langle T, N, R_S, S \rangle$

  $T = \{a,b\}$,\quad $N = \{S, A, B\}$;\quad $q_1 \to S$, $q_2 \to
  A$, $q_3 \to B$.
  
  $R = \{\begin{tabular}[t]{@{} l @{} l}
    $S \xrightarrow{1} aA$,\\
    $A \xrightarrow{0{,}4} aA$,\\
    $A \xrightarrow{0{,}5} bB$,\\
    $A \xrightarrow{0{,}1} b$,\\
    $B \xrightarrow{0{,}6} bB$,\\
    $B \xrightarrow{0{,}4} b$ & \}      
  \end{tabular}$
\end{ex}

\emph{Алгоритм определения $P(\alpha)$ для СА $S_S$}.  
$$L(S_S) = \{\langle \alpha, P(\alpha) \mid \alpha \in X^*,
\alpha=x_1,x_2,\ldots,x_m,\;P(\alpha) = \pi_0M(x_1)M(x_2) \ldots
M(x_m)\pi_F^T \}$$
где
\begin{tabular}[t]{l @{\;---\;} l @{}}
  $\pi_{0\langle n \rangle} = \langle \pi_{01}, \pi_{02}, \ldots,
  \pi_{0i}, \ldots, \pi_{0n} \rangle$ & начальное распределение
  состояний;\\
  $\pi_{F\langle n \rangle} = \langle \pi_{F1}, \pi_{F2}, \ldots,
  \pi_{Fj}, \ldots, \pi_{Fn} \rangle$ & конечное распределение
  состояний;\\
\end{tabular}\\
причём $\pi_{Fj} = 1$, если $q_j \in Q_F$.

\begin{ex}
  $S_S$ из примера на стр. \pageref{ex:KA}.
  \begin{list}{\arabic{N})}{\usecounter{N}}
  \item $\alpha = aab$;\quad $\pi_{0\langle 4 \rangle} = \langle 1, 0,
    0, 0 \rangle$;\quad $\pi_{F\langle 4 \rangle} = \langle 0, 0, 0, 1
    \rangle$;
    \begin{gather*}
      P(aab) = \langle 1,0,0,0 \rangle
      \begin{bmatrix}
        0 & 1 & 0 & 0\\
        0 & 0{,}4 & 0 & 0\\
        0 & 0 & 0 & 0\\
        0 & 0 & 0 & 0\\
      \end{bmatrix}
      \begin{bmatrix}
        0 & 1 & 0 & 0\\
        0 & 0{,}4 & 0 & 0\\
        0 & 0 & 0 & 0\\
        0 & 0 & 0 & 0\\
      \end{bmatrix}
      \begin{bmatrix}
        0 & 1 & 0 & 0\\
        0 & 0 & 0{,}5 & 0{,}1\\
        0 & 0 & 0{,}6 & 0{,}4\\
        0 & 0 & 0 & 0\\
      \end{bmatrix}
      \begin{bmatrix}
        0\\
        0\\
        0\\
        1\\
      \end{bmatrix}
      =\\
      = \langle 0,1,0,0 \rangle
      \begin{bmatrix}
        0 & 1 & 0 & 0\\
        0 & 0{,}4 & 0 & 0\\
        0 & 0 & 0 & 0\\
        0 & 0 & 0 & 0\\
      \end{bmatrix}
      \begin{bmatrix}
        0 & 1 & 0 & 0\\
        0 & 0 & 0{,}5 & 0{,}1\\
        0 & 0 & 0{,}6 & 0{,}4\\
        0 & 0 & 0 & 0\\
      \end{bmatrix}
      \begin{bmatrix}
        0\\
        0\\
        0\\
        1\\
      \end{bmatrix} = \\
      = \langle 0,0{,}4,0,0 \rangle
      \begin{bmatrix}
        0 & 1 & 0 & 0\\
        0 & 0 & 0{,}5 & 0{,}1\\
        0 & 0 & 0{,}6 & 0{,}4\\
        0 & 0 & 0 & 0\\
      \end{bmatrix}
      \begin{bmatrix}
        0\\
        0\\
        0\\
        1\\
      \end{bmatrix} = \\
      = \langle 0,0,0{,}2,0{,}04 \rangle
      \begin{bmatrix}
        0\\
        0\\
        0\\
        1\\
      \end{bmatrix} = 0{,}04
    \end{gather*}
  \item Определим $P(\alpha)$ для $G_S \sim S_S$\,:
    $$S \stackrel{1}{\Rightarrow} aA \stackrel{0{,}4}{\Rightarrow} aaA
    \stackrel{0{,}1}{\Rightarrow} aab \Rightarrow P(\alpha) = 1 \cdot
    0{,}4 \cdot 0{,}1 = 0{,}04\,.$$
  \end{list}
\end{ex}

  \Tree[.S a [.A a [.A b ] ] ]

%%%%%%%%%%%%%%%%%%%%%%%%%%%%%%%%%%%%%%%%%%%%%%%%%%%%%%%%%%%%%%%%%%%%%%%%%%%%%%%%
%%%%%%%%%%%%%%%%%%%%%%%%%%%%%%%%%%%%%%%%%%%%%%%%%%%%%%%%%%%%%%%%%%%%%%%%%%%%%%%%

\subsection{Восстановление (синтез) структурно-стохастических моделей (ССМ) функционирования БС}
\label{sec:synthesis}

Как указывалось при формулировке задач построения системы анализа ТМИ,
одной и центральных является проблема автоматического обучения системы
распознавания ТС.

При рассмотрении ССМ функционирования БС задача обучения системы
распознавания ТС формулируется как задача восстановления (синтеза) СГ.

%%%%%%%%%%%%%%%%%%%%%%%%%%%%%%%%%%%%%%%%%%%%%%%%%%%%%%%%%%%%%%%%%%%%%%%%%%%%%%%%

\subsubsection{Формулировка задачи восстановления СГ}

\begin{tabular}{@{} r @{\;---\;} p{.86\textwidth} @{}}
  $x_j$ & некоторый вычисляемый параметр ТС $x_j \in X$;\\
  $\bar{x}$ & значение параметра $\bar{x}$ в некоторый момент времени
  наблюдения распознавания;\\
  $\tilde{\bar{x}}$ & оценка значения параметра $x_j$, принимаемая по
  результатам его вычисления;\\
  $D_{\bar{x}_j}$ & дискретные значения, которые принимает параметр
  $x_j$ (область значений параметра $x_j$), $D_{\bar{x}_j}=
  \{a_{j1},a_{j2},\ldots,a_{jk},\ldots\}$;\\
  $|D_{\bar{x}_j}|$ & мощность (количество значений) параметра $\bar{x}_j$;\\
  $G_j$ & множество грамматик, каждая из которых порождает язык $L_j^k
  = L(G_j^k)$, соответствующий множеству значений тех параметров из
  $X$, с использованием которых вычисляется $\bar{x}_j$, $G_j =
  \{G_j^1,G_j^2,\ldots,G_j^k,\ldots\}$;\\ 
  $G_j^0$ & характеристическая грамматика соответствующей $G_j$.
\end{tabular}
\medskip

Для восстановления СГ $G_j^k$ используется отношение обучения такое,
что каждому языку $L_j^k$ (а значит, значению $\bar{x}_j = a_k \in
D_{\bar{x}_j}$) соответствует множество пар:
$$
\eta^k = \{\langle \alpha_i,m_i \rangle \mid \alpha_i \in L_j^k, \;
m_i\text{~--- кратность слова $\alpha$ в языке $L_j^k$}\}
$$

Тогда задача восстановления СГ $G_j^k$:

Дано:
\begin{tabular}[t]{l}
  $x_j \in X$,\\
  $a_{jk} \in D_{\bar{x}_j}$,\\
  $\eta^k = \{\langle \alpha_i,m_i \rangle\}$
\end{tabular}

Построить:
\begin{list}{\arabic{N})}{\usecounter{N}}
\item $G_j^0$~--- характеристическую грамматику;
\item $G_j^k$~--- стохастическую грамматику по значению $a_{jk}$.
\end{list}

%%%%%%%%%%%%%%%%%%%%%%%%%%%%%%%%%%%%%%%%%%%%%%%%%%%%%%%%%%%%%%%%%%%%%%%%%%%%%%%%

\subsubsection{Алгоритм синтеза СГ в условиях единственного
  значения $x_j$}

$$x_j: |D_{\bar{x}_j}| = 1$$

Рассмотрим СГ $G_S$ и обучающий язык $L_S$.
$$ L_S = \{\langle \alpha_l,m_l \rangle \}$$

$\hat{n}_{ij}(\alpha_l)$~--- случайная величина, равная числу
вхождений правила $A_i \to \beta_j$ в вывод цепочки $\alpha_l$ для
$G_S$.
$$ S \stackrel[G_S]{*}{\Rightarrow} \alpha_l$$

Оценка математического ожидания для $\hat{n}_{ij}(\alpha_l)$:
\begin{equation}
\tilde{M}[\hat{n}_{ij}] = \frac{\sum\limits_{\alpha_l \in L_S}m_l \cdot
  n_{ij}(\alpha_l)}{\sum\limits_{\alpha_l \in L_S}m_l}
\label{eq:M}
\end{equation}

В условиях согласованности СГ $G_S$ с $L_S$ оценка вероятности правил
подстановок:
\begin{equation}
  \tilde{P}_{ij} = \frac{\tilde{\bar{n}}_{ij}}{\sum\limits_{\langle i,j
      \rangle \in I_R^i}\tilde{\bar{n}}_{ij}} =
  \frac{\sum\limits_{\alpha_l \in L_S}m_l \cdot
    n_{ij}(\alpha_l)}{\sum\limits_{\langle i,j \rangle \in
      I_R^i}\sum\limits_{\alpha_l \in L_S}m_l \cdot n_{ij}(\alpha_l)}
\label{eq:Pij}
\end{equation}

где $I_R^i$~--- множество индексов правил в $R$ с левыми частями с $A_i$.

\begin{ex}
  См. пример на стр. \pageref{ex:KA}.

  $G_S^0\colon T = \{a,b\};\quad N = \{S,A,B\};\quad |L_S| = 100$
  \begin{center}
    \begin{tabular}{c l}
      1 & $S \xrightarrow{1} aA$,\\
      2 & $A \xrightarrow{0{,}4} aA$,\\
      3 & $A \xrightarrow{0{,}5} bB$,\\
      4 & $A \xrightarrow{0{,}1} b$,\\
      5 & $B \xrightarrow{0{,}6} bB$,\\
      6 & $B \xrightarrow{0{,}4} b$\\
    \end{tabular}
    \qquad
    \begin{tabular}{@{\quad} l @{\hspace{3em}} c @{\hspace{4em}} l @{\quad}}
      \hline\hline
      $\alpha_l$ & $m_l$ & $\{n_{ij}(\alpha_l)\}$\\\hline
      $a\ b\ b$ & 31 & 1,\ 3,\ 6\\
      $a\ b\ b\ b$ & 19 & 1,\ 3,\ 5,\ 6\\
      $a\ b$ & 15 & 1,\ 4\\
      $a\ a\ b\ b$ & 13 & 1,\ 2,\ 3,\ 6\\
      $a\ b\ b\ b\ b$ & 10 & 1,\ 3,\ 5,\ 5,\ 6\\
      $a\ a\ b$ & 6 & 1,\ 2,\ 4\\
      $a\ a\ a\ b\ b\ b$ & 3 & 1,\ 2,\ 2,\ 3,\ 5,\ 6\\
      $a\ a\ a\ b$ & 2 & 1,\ 2,\ 2,\ 4\\
      $a\ a\ a\ a\ b$ & 1 & 1,\ 2,\ 2,\ 2,\ 4\\\hline\hline
    \end{tabular}
  \end{center}

  \begin{center}
    \begin{tabular}{l l c c c}
      1 & $1\cdot(31 + 19 + 15 + 13 + 10 + 6 + 3 + 2 + 1)/100$ &
      $1$ & $1 / 1$ & $1{,}00$\\
      2 & $(1\cdot13 + 1\cdot6 + 2\cdot3 + 2\cdot2 +
      3\cdot1)/100$ & $0{,}32$ &  $0{,}32 / 1{,}32$ & $0{,}24$\\
      3 & $(1\cdot31 + 1\cdot19 + 1\cdot13 + 1\cdot10 + 1\cdot3)/100$
      & $0{,}76$ & $0{,}76/1{,}32$ & $0{,}58$\\
      4 & $(1\cdot15 + 1\cdot6 + 1\cdot2 + 1\cdot1)/100$ & $0{,}24$ &
      $0{,}24/1{,}32$ & $0{,}18$\\
      5 & $(1\cdot19 + 2\cdot10 + 1\cdot3)/100$ & $0{,}42$ &
      $0{,}42/1{,}18$ & $0{,}36$\\
      6 & $(1\cdot31 + 1\cdot19 + 1\cdot13 + 1\cdot10 + 1\cdot3)/100$
      & $0{,}76$ & $0{,}76/1{,}18$ & $0{,}64$\\
    \end{tabular}
  \end{center}
\end{ex}

%%%%%%%%%%%%%%%%%%%%%%%%%%%%%%%%%%%%%%%%%%%%%%%%%%%%%%%%%%%%%%%%%%%%%%%%%%%%%%%%

\subsubsection{Алгоритм синтеза СГ в условиях множества значений $x_j$}
\label{sec:synthesis_multitude}

$$x_j \colon |D_{\bar{x}_j} > 1$$

\noindentИмеем:\\
$D_{\bar{x}} = \{a_1,a_2,\ldots,a_k,\ldots\}$,\\
$L_S = L_{S_1} \cup L_{S_2} \cup \ldots \cup L_{S_k} \cup \ldots$,\\
$G_S = \{G_{S_1}, G_{S_2}, \ldots, G_{S_k}, \ldots\}$\\
$P(\bar{x} = a_k) = P(a_k)$~--- априорная вероятность того, что
$\bar{x} = a_k$, причём $\sum\limits_k^{|D_{\bar{x}}|} P(a_k) = 1$.\\
$P(\alpha_l / \bar{x}_j = a_k) = P(\alpha_l / a_k)$~--- условная
вероятность цепочки $\alpha_l$ при условии, что имеет место значение
$\bar{x} = a_k$.

Теперь в выражении \eqref{eq:M} для $\tilde{\bar{n}}_{ij}$ разделим
каждое слагаемое числителя дроби на знаменатель:
\begin{equation*}
  \tilde{\bar{n}}_{ij}^k = \sum_{\alpha_l \in L_{S_k}}\left[
    \frac{m_l}{\sum_{\alpha_f in L_S} m_f} \right]
  \cdot \tilde{n}_{ij}^k (\alpha_l)
\end{equation*}

Заметим, что дробь в этом выражении есть $P(\alpha_l)$. Однако, в
условиях множества значений $\bar{x}$ эта вероятность зависит от
значения $\bar{x} = a_k$. Следовательно, эта дробь определяется как
вероятность зависимого события:
$$P(\alpha_l / a_k)\cdot P(a_k)$$
тогда выражения \eqref{eq:M} и \eqref{eq:Pij} принимают вид:

\begin{equation}
  \tilde{\bar{n}}_{ij}^k = \sum_{\alpha_l \in L_{S_k}} P(\alpha_l /
  a_k) \cdot P(a_k) \cdot \tilde{n}_{ij}^k (\alpha_l)
\label{eq:nijk}
\end{equation}

\begin{equation}
  \tilde{P}_{ij}^k =
  \frac{\tilde{\bar{n}}_{ij}^k}{\sum\limits_{\langle i,j\rangle \in
      I_R^i} \tilde{\bar{n}}_{ij}^k}
\label{eq:Pijk}
\end{equation}

В выражениях \eqref{eq:nijk} и \eqref{eq:Pijk}:
\begin{itemize}
\item $P(\alpha_l / a_k)$~--- может быть определена как кратность
  слова $\alpha_l$, в языке $L_{S_k}$, пронормированная по этому
  языку: $P(\alpha_l / a_k) = \frac{m_l^k}{\sum\limits_{\alpha_i \in
      L_{S_k}} m_i^k}$
\item $P(a_k)$~--- может быть задана экспертом или может быть равна:
  $P(a_k) = \frac{1}{|D_{\bar{x}}|}$ (в условиях равновероятных
  значений $\bar{x} = a_k$)
\end{itemize}

%%%%%%%%%%%%%%%%%%%%%%%%%%%%%%%%%%%%%%%%%%%%%%%%%%%%%%%%%%%%%%%%%%%%%%%%%%%%%%%%

\subsubsection{Рекуррентный алгоритм восстановления СГ}
\label{sec:recurrent_recovery_algorithm}

Реально при обучении системы распознавания ТС имеется, как правило,
рекуррентный процесс обучения, организованный по шагам. В таких
условиях имеется последовательность обучающих множеств:
\begin{equation*}
  L_S^1 \subset L_S^2 \subset L_S^3 \ldots \subset L_S^n \subset
  L_S^{n+1} \subset \ldots,
\end{equation*}
где $n$~--- шаг обучения.

Имеем оценки для $\tilde{\bar{n}}_{ij}^k$ и $\tilde{P}_{ij}^k$ на
$n$-ном шаге обучения $\tilde{\bar{n}}_{ij}^k \Bigr|_n$ и
$ \tilde{P}_{ij}^k \Bigr|_n$. Тогда для $(n+1)$-ого шага получим (с
учётом \eqref{eq:nijk} и \eqref{eq:Pijk}):
\begin{equation*}
  \tilde{\bar{n}}_{ij}^k \Bigr|_{n+1} = \tilde{\bar{n}}_{ij}^k
  \Bigr|_n + \sum_{\alpha_l \in L_S^{n+1} \mathop{\backslash} L_S^n}
  P(\alpha_l \mathop{/} a_k) \cdot P(a_k) \cdot \tilde{n}_{ij}^k
  (\alpha_l)
\end{equation*}
\begin{equation*}
   \tilde{P}_{ij}^k \Bigr|_{n+1} = \frac{\tilde{\bar{n}}_{ij}^k
     \Bigr|_{n+1}}{\sum_{\langle i,j \rangle \in I_R^i}
     \tilde{\bar{n}}_{ij}^k \Bigr|_{n+1}}
\end{equation*}

При обучении возможен такой случай, когда априорные вероятности
значений $\bar{x} = a_k$ меняются от шага к шагу. Рассмотрим, как
изменяются оценки для $\tilde P_{ij}$.

$P(a_k) \Bigr|_n$~--- вероятность значения $\bar{x} = a_k$ на $n$-ном
шаге обучения.

$L_S^{n+1} \mathop{\backslash} L_S^n = \alpha^*$~--- обучающее
множество увеличилось на одно слово.

Тогда выражения для $\tilde{\bar{n}}_{ij}^k \Bigr|_{n+1}$:
\begin{equation*}
  \tilde{\bar{n}}_{ij}^k \Bigr|_{n+1} = \tilde{\bar{n}}_{ij}^k
  \Bigr|_n + P(\alpha^* \mathop{/} a_k) \cdot P(a_k)\Bigr|_{n+1} \cdot
  (\alpha^*)
\end{equation*}

По формуле Байеса имеем: $P(a_k \mathop{/} \alpha^*) =
\frac{P(\alpha^* \mathop{/} a_k) P(a_k)}{\sum_{m=1}^{|D_{\bar{x}}|}
  P(\alpha^* \mathop{/} a_m) P(a_m)}$ $\Rightarrow$

\begin{equation}
  P(a_k)\Bigr|_{n+1} = \frac{P(\alpha^* \mathop{/} a_k)
    P(a_k)\Bigr|_n}{\sum\limits_{m=1}^{|D_{\bar{x}}|} P(\alpha^* \mathop{/}
    a_m) P(a_m)}
\end{equation}

Это выражение, использованное вместе с ранее полученным, даёт довольно
мощный и гибкий алгоритм восстановления СГ, который может быть с
успехом использован при обучении системы распознавания ТС БС.

%%%%%%%%%%%%%%%%%%%%%%%%%%%%%%%%%%%%%%%%%%%%%%%%%%%%%%%%%%%%%%%%%%%%%%%%%%%%%%%%
%%%%%%%%%%%%%%%%%%%%%%%%%%%%%%%%%%%%%%%%%%%%%%%%%%%%%%%%%%%%%%%%%%%%%%%%%%%%%%%%

\subsection{Анализ информации (ТМИ) с использованием структурно-стохастических моделей (ССМ) функционирования БС}

В п.~\ref{sec:synthesis} мы рассмотрели методы восстановления ССМ с
использованием:
\begin{itemize}
\item характеристической грамматики, порождающей все возможные цепочки
  некоторого языка;
\item обучающего языка, являющегося подмножеством языка, порождаемого
  характеристической грамматикой.
\end{itemize}

Синтезированная ССМ в виде СГ позволяет с большой достоверностью
оценивать значения вычисляемых параметров ТС, а значит, и оценивать
сами ТС, в которых находится объект анализа.

Такое оценивание может быть произведено с помощью алгоритмов анализа
информации на основе \emph{анализа ССМ~--- СГ}.

В этом подразделе рассмотрим задачу анализа ТМИ с использованием СГ.

%%%%%%%%%%%%%%%%%%%%%%%%%%%%%%%%%%%%%%%%%%%%%%%%%%%%%%%%%%%%%%%%%%%%%%%%%%%%%%%%

\subsubsection{Формулировка задачи анализа ТМИ с использованием СГ}

Формальная постановка задачи анализа ТМИ формулируется с учётом задачи
синтеза СГ.

$x_j \in X$

$\bar{x}_j$~--- значение параметра $x_j$

$\tilde{\bar{x}}_j$~--- оценка значения параметра $x_j$

$D_{\bar{x}_j}= \{a_{j1},a_{j2},\ldots,a_{jk},\ldots\}$;

$|D_{\bar{x}_j}|$~--- мощность (количество значений) параметра
$\bar{x}_j$;

$G_j = \{G_j^1,G_j^2,\ldots,G_j^k,\ldots\}$~--- множество грамматик;

$\alpha^*$~--- некоторая входная цепочка в алфавите значений тех
параметров ТС $X$, с использованием которых определяется значение
параметра $x_j$ (или, в алфавите значений тех параметров ТС, от
которых зависит значение рассматриваемого параметра $x_j$.

Тогда задача анализа ТМИ, сводимая к задаче определения параметра ТС
$x_j$, формулируется как задача отнесения цепочек $\alpha^*$ к одному
из классов, каждый из которых задаётся множеством СГ $G_j$:

Дано: $\alpha^*$,\quad $G_j = \{G_{j1},G_{j2},\ldots,G_{jk},\ldots\}$.

Определить $G_{jm} \colon d_j^m = \max_{i \in I_{G_j}} \{d_j\}$,
т.\,е. определить ту СГ $G_{jm}$, что вероятность (мера)
принадлежности языку, порождаемому ею, будет максимальной.

Решение такой задачи проводится в два этапа:
\begin{description}
\item[на \emph{первом этапе}] проводится синтаксический анализ
  входного слова $\alpha^*$ для каждой СГ, соответствующей каждому
  значению параметра, т.,е. $x_j$.
\item[на \emph{втором этапе}] вычисляется апостериорная вероятность
  $P(a_{jk} \mathop{/} \alpha^*)$ для всех $k \in I_{G_j}$ и
  принимается решение о присвоении оценки значения параметра,
  т.,е. $\tilde{\bar{x}}_j$ той величине $a_{jm}$, которая обладает
  максимальной достоверностю, т.,е.
  \begin{equation*}
    a_{jm} = \arg \max_{a_{jk} \in D_{\bar{x}_j}} \{ P(a_{jk}
    \mathop{/} \alpha^*)\}
  \end{equation*}
\end{description}

%%%%%%%%%%%%%%%%%%%%%%%%%%%%%%%%%%%%%%%%%%%%%%%%%%%%%%%%%%%%%%%%%%%%%%%%%%%%%%%%

\subsubsection{Алгоритм анализа СГ в условиях единственности значения $x_j$}

\begin{equation*}
  |D_{xj}| = 1
\end{equation*}

Поскольку в случае одноэлементного множества значений параметров ТС,
когда $|D_{\bar{x}_j}| = 1$, мы имеем единственную СГ $G_j$, то
анализ входной цепочки $\alpha^*$ сводится к определению её
вероятности в СЯ $L(G_j^1)$, поскольку $P(a_1 \mathop{/} \alpha^*) =
P(\alpha^* \mathop{/} a_1) = P(\alpha^*)$.

Вероятность цепочки $\alpha^*$ определяется:
\begin{equation*}
  P(\alpha^*) = P(\alpha^*_V)\,,
\end{equation*}
где $\alpha_V^*$~--- слово вывода цепочки $\alpha^*$ в СГ $G_{j1}$.

\begin{rem}
  Слово вывода $\alpha_V^*$ цепочки $\alpha^*$ в ФГ $G$ определяется
  как слово в алфавите $I_R = \{r_1,r_2,\ldots\}$~--- номера правил
  вывода в $R$~--- множестве правил вывода
  \begin{equation*}
    \alpha_V^* = r_{j1}r_{j2}\ldots r_{jk_j}\,,
  \end{equation*}
  такое, что каждый элемент $\alpha_V^*$~--- есть номер того правила в
  $R$, которое применяется в выводе $S \stackrel[G]{*}{\Rightarrow}
  \alpha^*$, а порядок применения этих правил задаётся порядком
  следования $r_j$ в слове вывода $\alpha_V^*$.
\end{rem}

\begin{ex}[к замечанию]\ 

  \noindent$G \colon $ 

  $T = \{a,b\}$

  $N = \{S, A, B\}$
  
  $R = \{\begin{tabular}[t]{@{}c l @{} l}
    1 & $S \rightarrow aA$,\\
    2 & $A \rightarrow aA$,\\
    3 & $A \rightarrow bB$,\\
    4 & $A \rightarrow b$,\\
    5 & $B \rightarrow bB$,\\
    6 & $B \rightarrow b$ & \}      
  \end{tabular}$
  
  $I_R = \{1,2,3,4,5,6\}$.
  
  \begin{center}
    \begin{tabular}{@{\quad} l @{\qquad} l @{\quad}}
      \hline\hline
      $\alpha^*$ & $\alpha^*_V$\\\hline
      $a\ b$ & 1\ 4\\
      $a\ b\ b$ & 1\ 3\ 6\\
      $a\ a\ b\ b$ & 1\ 2\ 3\ 6\\\hline\hline
    \end{tabular}
  \end{center}
\end{ex}

Итак, вероятность цепочки $\alpha^*$ в СЯ $L(G_{jk})$ есть
\begin{equation*}
  \begin{split}
    P(\alpha^*) = P(\alpha^*_V) &= P(r_{j1}\ldots r_{jk_j})=\\
    &= P(r_{j1})P(r_{j2})\ldots P(r_{jk_j}) =\\
    &= \prod_{r_j \in \alpha_V^*} P(r_j).
  \end{split}
\end{equation*}

%%%%%%%%%%%%%%%%%%%%%%%%%%%%%%%%%%%%%%%%%%%%%%%%%%%%%%%%%%%%%%%%%%%%%%%%%%%%%%%%

\subsubsection{Алгоритм анализа СГ в условиях множества значений
  $x_j$}
\label{sec:analysis_single}


\begin{equation*}
  |D_{\bar{x}j}| = 1
\end{equation*}

В этом случае мы имеем множество СГ
\begin{equation*}
  G = \{G_1, G_2, \ldots, G_k, \ldots\}
\end{equation*}
(здесь и далее не будем указывать $j$).

Поэтому после проведения синтаксического анализа (грамматического
разбора) слова $\alpha^*$ по всем заданным характеристическим
грамматикам $G_k^0$, соответствующим СГ $G_k \in G$.

Далее проводится вычисление апостериорных вероятностей:
\begin{equation*}
  P(a_k \mathop{/} \alpha^*)
\end{equation*}
для всех $k \in I_G$ (по всему множеству $D_{\bar{x}j}$).

Для этого используется байесовский подход, в рамках которого:
\begin{equation*}
  P(a_k \mathop{/} \alpha^* = \frac{P(\alpha^* \mathop{/} a_k) P(a_k)}
  {\sum\limits_{i = 1}^{|D_{\bar{x}j}|} P(\alpha^* \mathop{/} a_i) P(a_i)}\,,
\end{equation*}
где \begin{tabular}[t]{@{} r @{\;---\;} p{.8\textwidth} @{}}
  $P(\alpha^* \mathop{/} a_k)$ & условные вероянтости принадлежности
  слова $\alpha^*$ языку $L(G_k)$, определяемые как в
  п.~\ref{sec:analysis_single};\\
  $P(a_k)$ & априорные вероятности $P(\bar{x}j = a_k)$ такие, что
  значение $\bar{x}j$ равно $a_k \in D_{\bar{x}j}$.
\end{tabular}

Величина $P(a_k)$, как и в случае восстановления СГ
(см. п.~\ref{sec:synthesis_multitude}) может быть задана экспертом, а
в условиях максимальной неопределённости~---
\begin{equation*}
  P(a_k) = \frac 1 {|D_{\bar{x}j}|}\text{~--- для $k = 1(1)
    \left|D_{\bar{x}j}\right|$}
\end{equation*}

\begin{ex}
  
\end{ex}

%%%%%%%%%%%%%%%%%%%%%%%%%%%%%%%%%%%%%%%%%%%%%%%%%%%%%%%%%%%%%%%%%%%%%%%%%%%%%%%%

\subsubsection{Рекуррентный алгоритм анализа СГ}

По аналогии с алгоритмом рекуррентного восстановления СГ
(см. п.~\ref{sec:recurrent_recovery_algorithm}), процесс анализа
входной цепочки в наиболее эффективном виде реализуется
рекуррентно~--- в процессе поступления символов входной цепочки.

\begin{tabular}[t]{@{} r @{\;---\;} p{.815\textwidth} @{}}
  $P(a_k)\bigr|_n$ & вероятность значения $\bar{x} = a_k$ на $n$-ном
  шаге анализа ($n = 0,1,2,\ldots$) (после анализа $n-1$ символов
  входной цепочки $\alpha^*$)\\
  $P(a_k)\bigr|_{n+1}$ & вероятность значения $\bar{x} = a_k$ на $n +
  1$-м шаге анализа (после получения $n$-ого символа цепочки
  $\alpha^*$.
\end{tabular}

Тогда $P(a_k)$ на последующем шаге определится через значение $P(a_k)$
на предыдущем шаге по следующей формуле:
\begin{equation*}
  P(a_k)\bigr|_{n+1} = \frac{P(\alpha^* \mathop{/} a_k) P(a_k)\bigr|_n}
  {\sum\limits_{i=1}^{|D_{\bar{x}j}|} P(\alpha^* \mathop{/} a_i) P(a_i)\bigr|_n}
\end{equation*}
где \begin{tabular}[t]{@{} r @{\;---\;} p{.8\textwidth} @{}}
  $P(\alpha^* \mathop{/} a_k)$ & определяется для каждого значения
  $a_k$ или, соответственно, для каждой СГ $G_k \in G$ через
  вероятность слова вывода.
\end{tabular}







%%% Local Variables: 
%%% mode: latex
%%% TeX-master: "lections"
%%% End: 

\end{document}