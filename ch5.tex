\subsection{Основные понятия теории формальных грамматик и языков}

\subsubsection{Определение формальной грамматики (ФГ) и языка}


\emph{Теория формальных грамматик и языков} является основным разделом
математической лингвистики~--- математической дисциплины,
ориентированной на изучение структуры естественных и искусственных
языков.

Эта теория возникла в 50-е годы в работах американского лингвиста
Хомского. Хомский исходил из потребностей естественного языка. Однако,
вскоре стало ясно, что методы его теории в не меньшей степени
применимы и к естественным языкам (в частности, информационный язык,
формируемый системой сбора ТМИ).

По характеру используемого математического аппарата теория формальных
грамматик и языков близка к теории алгоритмов и к теории автоматов.

Основным понятием в теории формальных грамматик является понятие
формальной грамматики.

\begin{defin}
  \emph{Формальной грамматикой} $G$ называется кортеж (четвёрка)
  \begin{equation}
    \label{eq:G}
    G = \left<T, N, R, S\right>,
  \end{equation}
  
  {\hangindent=4em
    {\noindent где $T$~--- конечное множество терминальных символов, алфавит
      терминальных символов, терминалов;}
    
    $N$~--- конечное множество нетерминальных (вспомогательных)
    символов; $T \cap N = \varnothing$
    
    $R$~--- конечное множество упорядоченных пар $(\alpha, \beta)$,
    имеющих вид $\alpha \to \beta$, и называемых правилами подстановки,
    вывода, продукциями грамматики.
    
    $S \in N$~--- начальный символ (аксиома).
  }
\end{defin}

\begin{rem}
  Если $T$~--- терминальный алфавит, то $T^{*}$~--- множество слов
  в алфавите~$T$.
\end{rem}

\begin{rem}
  Будем в дальнейшем обозначать:
  \begin{itemize}
  \item элементы алфавита $T$: $a, b, c, \ldots$
  \item элементы алфавита $N$: $A, B, C, \ldots$
  \end{itemize}
\end{rem}

\begin{defin}
  Продукции грамматики $G$ определим как:
  \begin{equation}
    \label{eq:rule}
    R = \{\alpha\to\beta \mid \alpha \in (N \cup T)^{*} \times N \times (N
    \cup T)^{*},\; \beta \in (N \cup T)^{*}\}
  \end{equation}
\end{defin}

\begin{rem}
  Свойства отношения "$\to$" (отношения порядка):
      \begin{itemize}
      \item рефлексивность,
      \item антисимметричность,
      \item транзитивность.
      \end{itemize}
\end{rem}

\begin{defin}[выводимости слов]
  Будем говорить, что
  \begin{equation}
    \label{eq:deduction}
    \gamma \stackrel[G]{}{\Rightarrow} \varepsilon\qquad (\gamma,
    \varepsilon \in (N \cup T)^*)\,,
  \end{equation}
  если $\gamma = \sigma_1 \alpha \sigma_2$, $\varepsilon = \sigma_1
  \beta \sigma_2$ ($\sigma_1, \sigma_2 \in (N \cup T)^*$),
  и если подстановка $\alpha \to \beta$ входит в схему~$R$ грамматики~$G$.
\end{defin}

\emph{Содержательно} $\gamma \Rightarrow \varepsilon$ означает, что
цепочка $\varepsilon$ выводима из цепочки $\gamma$ путём подстановки в
цепочку $\gamma$ вместо цепочки $\alpha$ цепочки $\beta$, причём
подстановка входит в множество $R$: $$\alpha \to \beta\,.$$

\begin{defin}
  Будем говорить, что $$\gamma \stackrel[G]{*}{\Rightarrow}
  \varepsilon \; (\gamma,\varepsilon \in (N \cup T)^{*})\,,$$
  если существует $\gamma_0 = \gamma$ и $\gamma_n = \varepsilon$,
  такие, что $$\gamma_0 \Rightarrow \gamma_1 \Rightarrow \gamma_2
  \Rightarrow \ldots \Rightarrow \gamma_{n-1} \Rightarrow \gamma_n\,.$$
\end{defin}

Последовательность $\gamma_0, \gamma_1, \ldots, \gamma_n$ называется
\emph{выводом} длины $n$.

В ФС: \emph{выводимость} (непосредственная выводимость): $\alpha
\vdash \beta$ ($\frac{\phantom{0}\alpha\phantom{0}}{\beta}$)~--- правило
вывода, а выводимость $[\alpha \Rightarrow \beta] \Leftrightarrow
[\alpha = \varepsilon_1 \vdash \varepsilon_2 \vdash \varepsilon_3
\vdash \ldots \vdash \varepsilon_n = \beta]$

\begin{defin}
  \emph{Языком} $L(G)$, порождённым грамматикой $G$, называется
  следующее подмножество множества $T^*$:
  \begin{equation}
    \label{eq:lang}
    L(G) = \{\alpha \mid S \stackrel[G]{*}{\Rightarrow} \alpha,\;
    \alpha \in T^*\}\,,
  \end{equation}
  т.\,е. множество всех слов $\alpha \in T^*$, состоящих только из
  терминальных символов, выводимо в грамматике из начальной аксиомы $S$.
\end{defin}

\begin{defin}
  Количество символов, входящих в слово $\alpha \in L(G)$, будем
  называть его \emph{длиной} и обозначать $|\alpha|$.
\end{defin}

\begin{ex}
  \label{ex:G}
  $G = \langle T,N,P,S\rangle$, где $T=\{a,+,*\}$, $N = \{S,A\}$,
  $P$:
  \begin{enumerate}
  \item $S \to S*S$;
  \item $S \to A+A$;
  \item $A \to A*A$;
  \item $A \to a$.
  \end{enumerate}
  Вывод:
  $S \stackrel{2}{\Rightarrow} A+A \stackrel{4}{\Rightarrow} A*A+A
  \stackrel{5}{\Rightarrow} a*A+A \stackrel{5}{\Rightarrow} a*a+A
  \stackrel{5}{\Rightarrow} a*a+a$.

Язык $L(G)$~--- множество правильно построенных форм (ППФ) с
использованием операций $+$, $*$ и переменной $a$.
\end{ex}

\begin{defin}
  Помеченное дерево $D$ называется \emph{деревом (графом) вывода}
  грамматики $G$, если:
  \begin{enumerate}
  \item Корень дерева помечен символом аксиомы $S$;
  \item Если $D_1,D_2$~--- поддеревья дерева $D$, то корень каждого
    поддерева $D_i$ помечен:
    \begin{itemize}
    \item символом $A_i\;(A_i \in N)$, если дерево $D_i$ имеет больше
      одной вершины;
    \item символом $X_i\;(X_i \in T)$, если дерево $D_i$ состоит из
      единственной вершины.
    \end{itemize}
  \end{enumerate}
\end{defin}

\begin{rem}
  $D_i$ является деревом вывода в грамматике $G_i = \langle
  T,N,D,X_i\rangle$ (здесь $X_i$~--- есть аксиома дерева $D_i$).
\end{rem}

\begin{ex}
  Для грамматики $G$ (из прошлого примера на стр. \pageref{ex:G})
  вывод цепочки $a*a+a$ представлен на рис.~\ref{fig:deduction}.
  
  \begin{figure}[h]

    \Tree [.S [.A [.A a ] * [.A a ] ] + [.A a ] ]
    
    \caption{Вывод цепочки $a*a+a$}
    \label{fig:deduction}
  \end{figure}
  
\end{ex}

\begin{defin}
  Грамматика называется \emph{однозначной}, если каждое слово языка
  $L(G)$ имеет только одно дерево вывода. В противном случае
  грамматика называется \emph{неоднозначной}.
\end{defin}

\begin{rem}
  Свойство однозначности характеризует \emph{грамматику}, а не язык,
  поскольку один и тот же язык может быть описан различными
  грамматиками, среди которых могут быть как однозначные, так и
  неоднозначные.
\end{rem}

\begin{ex}
  Для той же грамматики (см. пример на с.~\pageref{ex:G}) и цепочки
  $a*a+a$:

  \begin{figure}[h]
    \centering
    \qtreecenterfalse

    (1)
    \Tree [.S [.A [.A a ] * [.A a ] ] + [.A a ] ]
    \hskip .1\textwidth
    (2)
    \Tree [.S\1 [.S [.A a ] ] * [.S [.A a ] + [.A a ] ] ]

    \caption{Два варианта вывода цепочки $a*a+a$}
  \end{figure}
\end{ex}

%%%%%%%%%%%%%%%%%%%%%%%%%%%%%%%%%%%%%%%%%%%%%%%%%%%%%%%%%%%%%%%%%%%%%%%%%%%%%%%%

\subsubsection{Классификация формальных грамматик и языков}

Формальные грамматики (ФГ) делятся на 3 категории:
\begin{itemize}
\item распознающие ФГ;
\item порождающие ФГ;
\item преобразующие ФГ.
\end{itemize}

\begin{defin}
  ФГ называется \emph{распознающей}, если для любой рассматриваемой
  цепочки она позволяет ответить на вопрос: является ли эта цепочка
  правильной ($\in L(G)$) или нет, и в случае положительного ответа
  даёт описание структуры (строение) этой цепочки.

  ФГ называется \emph{порождающей}, если она позволяет строить любую
  правильную цепочку, давая при этом описание её структуры, и не
  позволяет строить ни одной неправильной цепочки.

  ФГ называется \emph{преобразующей}, если для любой правильно
  построенной цепочки она позволяет строить \emph{отображение} её в
  виде цепочки, задавая при этом порядок реализации этого отображения.
\end{defin}

Рассмотрим класс \emph{порождающих} грамматик.

Основатель теории ФГ Н.\,Хомский провёл следующую классификацию (по
виду правил):


\begin{table}[!hp]
  \centering
  \begin{tabular}[!hp]{|c|p{.39\textwidth}|p{.4\textwidth}|}
    \hline
    Тип ФГ & Название ФГ & Вид порождающих правил ${P = \{\alpha \to
    \beta\}}$ \\
    \hline
    0 & Без ограничений & $\alpha \in (T \cup N)^* \times N \times (T
    \cup N)^*$, ${\beta \in (T \cup N)^*}$\\
    \hline
    1 & Контекстно-зависимые (непосредственно составляющих,
    неукорачивающие) & $\alpha \in (T \cup N)^* \times N \times (T
    \cup N)^*$, ${\beta \in (T \cup N)^*}$, $|\alpha| \leqslant
    |\beta|$\\
    \hline
    2 & Контекстно-свободные & $\alpha \in N$, $\beta \in (T \cup
    N)^*$ \\
    \hline
    3 & Регулярные (праволинейные) & $\alpha \in N$, $\beta \in T
    \times N$, $\beta \in T$\\
    \hline
  \end{tabular}
  \label{tab:chomsky}
\end{table}

\begin{ex}
  Правила вывода для разных типов языков
  \begin{center}
    \begin{tabular}[!hp]{p{11em}p{11em}p{8em}}
      $S \to aB$ & $S \to ABa$  & $AS \to ABC$ \\
      $B \to bC$ & $A \to cAAc$ & $Ab \to bbCb$\\
      $C \to b$  & $B \to a$    & $B  \to b$   \\
                 &              & $C  \to c$   \\[.5em]
      Тип 3 (Автоматная) & Тип 2 (КС) & Тип 1 (НС)\\
    \end{tabular}
  \end{center}
\end{ex}

\begin{rem}
  ФГ является одним из видов формальной системы с:
  \begin{enumerate}
  \item алфавитом $A = T \cup N$ (или языком $L(G)$);
  \item аксиомой $S \in N$;
  \item правилами вывода $P = \{\alpha \to \beta\}$.
  \end{enumerate}
\end{rem}

\begin{defin}
  \emph{ФГ (ФЯ) типа 0}~--- такие ФГ, в которых не накладывается
  никаких ограничений на правила подстановок. Эти ФГ позволяют
  порождать любые \emph{рекурсивно-перечислимые множества} и
  эквивалентны по мощности МТ, частично-рекурсивным функциям.
\end{defin}

\begin{defin}
  \emph{ФГ (ФЯ) типа 1} или ФГ НС, или КЗ, или \emph{неукорачивающие
    ФГ}~--- такие ФГ, в которых каждое правило $\alpha \to \beta$
  удовлетворяет соотношению $|\alpha| \leqslant |\beta|$ и каждое
  правило подстановки имеет вид (или может быть приведено к виду)
  \begin{equation*}
    \sigma_1 A \sigma_2 \to \sigma_1 \gamma \sigma_2\,,\quad A \in N,\;\;
    \sigma_1,\sigma_2 \in (T \cup N)^*,\;\; \gamma \in (T \cup N)^+
  \end{equation*}
  ФЯ типа 1~--- \emph{рекурсивное множество}.
\end{defin}

\begin{rem}
  В ФГ типа 1 могут быть правила $AB \to BA$
  ($\boxed{\phantom{A}}$~--- контексты, ${A',B' \in N}$~--- новые
  нетерминалы):
  \begin{equation*}
    AB \to BA \Leftrightarrow A\boxed{B} \to A'\boxed{B}
    \Leftrightarrow \boxed{A'}B \to \boxed{A'}B' \Leftrightarrow
    A'\boxed{B'} \to B\boxed{B'} \Leftrightarrow \boxed{B}B' \to
    \boxed{B}A\ldotp
  \end{equation*}

\end{rem}

\begin{defin}
  \emph{ФГ (ФЯ) типа 2}~или КС-грамматики~--- такие ФГ, в которых
  каждое правило подстановки имеет вид
  \begin{equation*}
    A \to \gamma,\quad \text{где} A \in N,\;\; \gamma \in (N \cup T)^*\,.
  \end{equation*}
  ФЯ типа 2~--- \emph{рекурсивное множество}.
\end{defin}

\begin{defin}
  \emph{ФГ (ФЯ) типа 3} (регулярные, автоматные)~--- такие ФГ, в
  которых каждое правило имеет следующий вид:
  $$
  \begin{cases}
    A \to aB & \text{или} \\
    A \to a, & \text{где $A,B \in N$, $a \in T$}\\
  \end{cases}
  $$
  ФЯ типа 3~--- \emph{регулярные события}.
\end{defin}

%%%%%%%%%%%%%%%%%%%%%%%%%%%%%%%%%%%%%%%%%%%%%%%%%%%%%%%%%%%%%%%%%%%%%%%%%%%%%%%%

\subsubsection{Формальные свойства грамматик}

При решении задач, связанных с грамматиками (с распознаванием слов по
грамматикам, с построением грамматик, с преобразованием грамматик,
\ldots) мы всегда сталкиваемся с проблемами их разрешимости: то есть
существует ли вообще алгоритм, позволяющий решать ту или иную
задачу. В этом случае нужно обратиться к изучению \emph{формальных
  свойств грамматик}, или, к \emph{алгоритмическим проблемам теории
  формальных грамматик}. Перечислим некоторые из них ($+$~---
разрешимая проблема, $-$~--- неразрешимая):


\begin{longtable}{|p{.6\textwidth}|p{.05\textwidth}|p{.05\textwidth}|p{.05\textwidth}|p{.05\textwidth}|}
  \hline \multirow{2}*{Название проблемы} & \multicolumn{4}{c|}{Тип
    грамматики}\\
  \cline{2-5}
  & 3 & 2 & 1 & 0\\
  \hline \endhead 1. Пуст ли язык, порождённый данной грамматикой
  ($L_G =
  \varnothing$)? & $+$ & $+$ & $-$ & $-$ \\
  \hline 2. Бесконечен ли язык, порождённый данной грамматикой ($|L_G|
  =
  \infty$)? & $+$ & $+$ & $-$ & $-$ \\
  \hline 3. Включает ли язык, порождённый данной грамматикой, все
  слова в
  алфавите $T$ ($L_G = T^*$)? & $+$ & $-$ & $-$ & $-$ \\
  \hline 4. Составляет ли язык, порождённый данной грамматикой,
  подмножество языка, порождаемого другой ($L_{G1} \subseteq
  L_{G2}$)? & $+$ & $-$ & $-$ & $-$ \\
  \hline 5. Порождают ли две грамматики один и тот же язык ($L_{G1} =
  L_{G2}$)? & $+$ & $-$ & $-$ & $-$ \\
  \hline 6. Пусто ли \emph{пересечение} языков, порождаемых двумя
  грамматиками ($L_{G1} \cap L_{G2} = \varnothing$)? & $+$ & $-$ & $-$
  & $-$
  \\
  \hline 7. Для $\forall \alpha, \beta \in (T \cup N)^*$ выводимо ли
  $\alpha \stackrel[G]{}{\Rightarrow} \beta$ ($S
  \stackrel[G]{}{\Rightarrow} \gamma$; $\gamma \in T$)? & $+$ & $+$ &
  $+$ & $-$ \\
  \hline 8. Есть ли в языке, порождённом данной грамматикой, слово,
  выводимое более, чем один раз (т.\,е. является ли грамматика
  однозначной)? & $+$ & $-$ & $-$ & $-$ \\
  \hline 9. Существует ли однозначная грамматика того же языка,
  порождающая такой же язык? & $+$ & $-$ & $?$ & $+$ \\
  \hline
\end{longtable}

%%%%%%%%%%%%%%%%%%%%%%%%%%%%%%%%%%%%%%%%%%%%%%%%%%%%%%%%%%%%%%%%%%%%%%%%%%%%%%%%
%%%%%%%%%%%%%%%%%%%%%%%%%%%%%%%%%%%%%%%%%%%%%%%%%%%%%%%%%%%%%%%%%%%%%%%%%%%%%%%%
%%%%%%%%%%%%%%%%%%%%%%%%%%%%%%%%%%%%%%%%%%%%%%%%%%%%%%%%%%%%%%%%%%%%%%%%%%%%%%%%

\subsection{Задача анализа ТМИ как задача распознавания образов (ТС)}

\subsubsection{Вводные определения}

Существует \emph{большое многообразие} задач \textbf{\emph{анализа
    ТМИ}}, а также существует множество подходов к решению этих
задач. В наиболее общей форме задача анализа ТМИ может быть
сформулирована как задача \emph{распознавания образов} (теория
распознавания образов). Причём в качестве образа рассматривается
\emph{ТС}.

\begin{defin}
  \emph{ТС} ОУ (объекта анализа) будем называть совокупность
  изменяющихся в процессе производства испытаний, эксплуатации свойств
  (как есть) ОУ, характеризующих его функциональную пригодность в
  заданных условиях применения.
\end{defin}

ТС определяется путём оценивания \emph{параметров ТС}, множество
значений которых образует некоторое \emph{пространство параметров
  ТС}. Из этого следует, что определение (оценивание) ТС в процессе ИТО~---
заключается в:
\begin{enumerate}
\item[1)] \emph{указании некоторой точки};
\item[2)] отнесении её к \emph{определённой области} в
  \emph{пространстве параметров ТС}.
\end{enumerate}

Параметры ТС:
\begin{itemize}
\item измеряемые $X_и$;
\item вычисляемые $X_в$.
\end{itemize}

\begin{defin}
  \emph{Измеряемыми параметрами ТС} (ИПТС) являются представимые в
  виде значений ТМП показатели (характеристики) свойств ОУ.

  Совокупность измеряемых ПТС образует пространство ИПТС.

  \emph{Вычисляемыми параметрами ТС} (ВПТС) являются такие показатели
  (характеристики) свойств ОУ, которые могут быть вычислены по
  различным алгоритмам с использованием значений измеряемых параметров ТС.
\end{defin}

\begin{defin}
  \emph{Целью \textbf{анализа} ТМИ} как процесса является получение
  обобщённых оценок совокупности ПТС (с учётом конкретных целей
  применения ЛА на различных этапах его функционирования), значения которых в явном виде указывают:
  \begin{itemize}
  \item либо степень работоспособности ОУ;
  \item либо место и вид возникшей на борту неисправности;
  \item либо являются оценками прогнозируемых процессов и явлений с
    заданной точностью и интервалом прогноза;
  \item и т.\,п.
  \end{itemize}
\end{defin}

\emph{Получить цель анализа}~--- это значит указать точку в
пространстве ПТС, характеризующую ТС ОУ.

\emph{Цель анализа}~--- задаётся экспертом или лицом, принимающим
решение и осуществляющим управление ОУ.

Из этого следует, что анализ ТМИ~--- есть процесс получения оценок
параметров ТС, являющихся элементами цели анализа, вместе с оценками
показателей степени доверия полученным результатам.

Цель анализа может соответствовать точке как во всём пространстве ПТС,
так и в его подпространстве ИПТС. Разделяя ИТО на обработку и анализ,
мы тем самым считаем (содержательно) обработку одним из этапов
анализа.

В частности, весь анализ в отдельных случаях может заключаться только
в обработке ТМИ,~--- так будет тогда, когда целью анализа являются
элементы множества ИПТС.

Таким образом, \emph{задачей анализа ТМИ} является формирование и
отнесение точки в пространстве ТС к той или иной области, которой
сопоставляется, например, допустимое управляющее воздействие.

%%%%%%%%%%%%%%%%%%%%%%%%%%%%%%%%%%%%%%%%%%%%%%%%%%%%%%%%%%%%%%%%%%%%%%%%%%%%%%%%

\subsubsection{Формальная постановка задачи распознавания ТС}

Предварительно введём ряд базовых множеств.

$\Omega = \{\omega\}$~--- множество образов (ТС), подлежащих
распознаванию (классификации); $\Omega$~--- счётное множество.

$\Omega' = \{\omega\}$~--- обучающее множество образов (ТС);
$|\Omega'| = k$, $\Omega$ конечно.

$X = \{n\}$~--- множество параметров ТС; $X_и \cup X_в$.

$G = \{G_i\}$~--- множество классов образов (классов ТС).

\begin{figure}[h]
  \centering
  $$
  \xymatrix{
    \Omega'\ar[rr]^{\varphi'}\ar[ddrr]_(0.7){\eta} & & X_и\ar[d]^{\psi} \\
    & & X_в\ar[d]^{\delta} \\
    \Omega\ar[uu]_{\xi}\ar[rr]^{\zeta}\ar[uurr]^(0.7){\varphi}  & & G \\
  }
  $$
  \caption{Коммутативная диаграмма}
  \label{fig:cd}
\end{figure}

\begin{longtable}{p{0em}p{0em}c@{\;:\;}c@{\;$\to$\;}lp{26em}}
  $\mathrm{H}$ && $\eta$ & $\Omega'$ & $G$ & отношение обучения\\
  
  &\multirow{2}*{$\arraycolsep=0em
    \left\{\begin{array}{c}\\\\\end{array}\right.$} & $\varphi$ &
  $\Omega$ &
  \multirow{2}*{$\arraycolsep=0em
    \left.\begin{array}{l}X_и\\X_и\end{array}\right\}$} &
  \multirow{2}*{отношения наблюдения (измерения)} \\
  
  && $\varphi'$ & $\Omega'$ \\ 
  
  && $\psi$ & $X_и$ & $X_в$ & отношение вычисления вычислимости \\
    
  && $\delta$ & $X_в$ & $G$ & отношение интерпретации \\
  
  $\Xi$        && $\xi$ & $\Omega$ & $\Omega'$ & отношение обобщения \\
  $\mathrm{Z}$ &&  $\zeta$ & $\Omega$ & $G$ & отношение классификации
\end{longtable}

В теории формальных грамматик в качестве $\omega \in \Omega$
рассматривается слово (цепочка) $\alpha$ в некотором языке,
т.\,е. $\omega = \alpha$, $\Omega = L$, $\Omega' = L'$~--- обучающий
язык, $G_i$~--- грамматика, описывающая некоторый класс (множество)
слов $\alpha \subset L_i$.

\noindent\emph{Задача распознавания ТС}:

Дано: $\alpha \in L$.

Определить: принадлежность $\alpha \in \{L_1, L_2, \ldots\}$

\noindent Задача обучения системы распознавания

Дано: $\eta$~--- отношение обучения:

$\eta = \{ \langle\alpha,G_i\rangle \mid i \in I_G \}$

Определить: $G_i$, $\forall i \in I_G$

Как на этапе обучения, так и на этапе классификации процесс вычислений
связан с: 1) \emph{выбором} и 2) \emph{реализацией} каких-либо алгоритмов.
 

$A$~--- множество алгоритмов, позволяющих реализовать отношение
классификации $\zeta$ (или $\psi$): $A = \{a_i\}$.

\noindent $X_{a_i}^+$~--- множество входных операндов для $a \in A$;

\noindent $X_{a_i}^-$~--- множество выходных операндов для $a \in A$.

Тогда задача распознавания ТС разбивается на 2 подзадачи:
\begin{enumerate}
\item[1)] задача обучения;
\item[2)] задача непосредственно распознавания.
\end{enumerate}

\begin{enumerate}
\item Задача обучения.

  Дано: $X$, $\Omega'$, $c_i \in C'$, $G^i \stackrel[*]{}{\subset} G$,
  $\eta$~--- отношение обучения.
  \begin{enumerate}
  \item Построить: $A^i = \{a\}$.
  \item Выбрать $a^{op\,r} \in A^i$, $\tau_{a^{op\,r}} = \min\limits_{r \in
      I_a} \{\tau_{a^r}\}$~--- операторный анализ.
  \end{enumerate}

\item Задача классификации.

  Дано: $\omega \in \Omega$.

  Определить: $G^0 \colon d_{G^0} = \max\limits_{i\in{}I_G}
  \{d_i\}$~--- достоверный анализ.
\end{enumerate}

%%%%%%%%%%%%%%%%%%%%%%%%%%%%%%%%%%%%%%%%%%%%%%%%%%%%%%%%%%%%%%%%%%%%%%%%%%%%%%%%
%%%%%%%%%%%%%%%%%%%%%%%%%%%%%%%%%%%%%%%%%%%%%%%%%%%%%%%%%%%%%%%%%%%%%%%%%%%%%%%%
%%%%%%%%%%%%%%%%%%%%%%%%%%%%%%%%%%%%%%%%%%%%%%%%%%%%%%%%%%%%%%%%%%%%%%%%%%%%%%%%

\subsection{Автоматные грамматики и языки}

Как было указано ранее, самым простым по мощности и выразительной
способности является грамматика $G$ и порождаемый ею язык $L(G)$ типа
3~--- автоматный (регулярный) язык и грамматика.

\begin{rem}
Вспомним определение автоматной грамматики:

  ФГ~--- типа 3, если каждое её правило имеет вид: $A \to aB$, $A \to
  a$, где $A,B \in N$, $a \in T$.
\end{rem}

Мы говорили о праволинейной (правосторонней) ФГ $G$:
\begin{itemize}
\item что такое праволинейная ФГ?
\end{itemize}

Можно показать, что $L(G_{пр} \Leftrightarrow L(G_{лев})$, у которых
$G_{пр} \colon A \to aB \Leftrightarrow G_{лев} \colon A \to Ba$.

Доказать: $L(G_{пр}) \equiv L(G_{лев})$

Существует связь между КА и речевыми грамматиками. Это утверждают
следующие теории, которые к тому же указывают и путь получения \emph{КА по
автоматной грамматике} и \emph{автоматной грамматики по КА}:

\begin{theorem}
  Если $G = \langle T,N,P,S \rangle$~--- грамматика типа 3 (РГ), то
  существует КА \underline{без ...}: $S_{КА} = \langle
  A,Q,\delta,q_0,F \rangle$ такой, что допускаемое им множество
  входных слов совпадает в точности с множеством слов языка $L(G)$,
  причём:
  \begin{enumerate}
  \item $T \to A$;
  \item $N \to Q$;
  \item $S \to q_0$\quad $(S \in N, q_0 \in Q)$;
  \item $[B \to aD] \to [\delta(a,B) = D]$;
  \item $[B \to a] \to [\delta(a,B) \in F]$, $F$~--- множество
    финальных состояний КА $S_{КА}$;
  \end{enumerate}
  и наоборот.
\end{theorem}

\begin{theorem}
  Класс языков, допускаемых КА, в точности совпадает с классом
  регулярных языков.
\end{theorem}

\begin{ex}
  $G\colon P = \{S \to aB, B \to bS, B \to b\}$.
  \begin{figure}[h]
    \centering
    \begin{tabular}{c|cc}
      & a & b\\\hline
      S & B & $-$\\
      B & $-$ & S
    \end{tabular}
    \large\qquad$\xymatrix{S\ar@/^/[rr]|a & &
      \boxed{B}\ar@/^/[ll]|b}$
  \end{figure}
\end{ex}

%%%%%%%%%%%%%%%%%%%%%%%%%%%%%%%%%%%%%%%%%%%%%%%%%%%%%%%%%%%%%%%%%%%%%%%%%%%%%%%%

\subsubsection{Восстановление (синтез) ФГ типа 3}

В теории анализа ТМИ, как уже указывалось, рассматриваются 2 проблемы,
решение которых используется при распознавании ТС:
\begin{enumerate}
\item синтез ФГ (на этапе обучения);
\item анализ ФГ (ФЯ)~--- на этапе распознавания ТС.
\end{enumerate}

Рассмотрим 1-ую из них.

\begin{rem}
  В силу двух последних теорем, для синтеза ФГ типа 3 достаточно
  построить КА, допускающий заданный язык, а затем по этому КА
  построить ФГ.
\end{rem}

Путь синтеза ФГ типа 3.
$\boxed{L \to S \to S^0 \to G} \Leftrightarrow \boxed{L \to S \to S^0
  \to G}$

\begin{ex}
  $L = \{a^mb^nc^kd \mid m,n,k \in \mathbb{N}\}$
\end{ex}

%%%%%%%%%%%%%%%%%%%%%%%%%%%%%%%%%%%%%%%%%%%%%%%%%%%%%%%%%%%%%%%%%%%%%%%%%%%%%%%%

\subsubsection{Анализ ФГ типа 3}

Существует два подхода к анализу ФГ типа 3:
\begin{enumerate}
\item алгоритмы анализа, базирующиеся на алгоритмах анализа КА;
\item алгоритмы анализа, базирующиеся на непосредственно
  грамматическом выводе.
\end{enumerate}

\emph{1-й подход} заключается в следующем:
$$
\xymatrix{
  G\ar[r] & S_{КА}\ar[r] & \{\alpha \in L(G),\: \alpha \notin L(G)\}\\
  & \alpha \ar[u]
}$$

\emph{2-й подход}:
$$
\xymatrix{
  G\ar[r] & \{\alpha \in L(G),\: \alpha \notin L(G)\}\\
  \alpha \ar[u]
}$$

\begin{ex}
  \ 
  
  $\begin{aligned}[t]
    G\colon P &= \{S \to aS \mid aA,  \\
    \phantom{G\colon P } & \phantom{\mathrel{{} = \{}}
    \lefteqn{A}\phantom{S} \to bA \mid bS \mid b\}\\
    \alpha_1 &= aaabbbaab \\
    \alpha_2 &= aabbab
  \end{aligned}
  $

  Эта ФГ~--- недетерминированная, но не является неоднозначной.
\end{ex}

%%%%%%%%%%%%%%%%%%%%%%%%%%%%%%%%%%%%%%%%%%%%%%%%%%%%%%%%%%%%%%%%%%%%%%%%%%%%%%%%
%%%%%%%%%%%%%%%%%%%%%%%%%%%%%%%%%%%%%%%%%%%%%%%%%%%%%%%%%%%%%%%%%%%%%%%%%%%%%%%%
%%%%%%%%%%%%%%%%%%%%%%%%%%%%%%%%%%%%%%%%%%%%%%%%%%%%%%%%%%%%%%%%%%%%%%%%%%%%%%%%

\subsection{КС-грамматики и языки}

\noindent6. Рейуорд--Смит, В.\,Дж Теория формальных языков. Вводный курс /
Пер. с англ. А.\,Кузьмина, под ред. И.Г.\,Шестакова. М.: Радио и
связь, 1988~--- 128 с.

%%%%%%%%%%%%%%%%%%%%%%%%%%%%%%%%%%%%%%%%%%%%%%%%%%%%%%%%%%%%%%%%%%%%%%%%%%%%%%%%

\subsubsection{Нормальные формы КС грамматик}

Правила КС-грамматик имеют вид: 
$A \to \alpha$, где $A\in N,\;\alpha \in (T \cup N)^*$.

Такое ограничение является очень слабым. В силу этого КС-грамматики и
языки имеют очень большую порождающую мощность и поэтому большей
частью используются при распознавании ТС, в других областях прикладной
математики.

Однако это \emph{достоинство} (мощность КС-грамматик), определяемое
существенным ограничением $A \to \alpha$, является
\emph{недостатком} для изучения свойств таких КС грамматик. Вот почему
возникли различные \emph{нормальные формы} при задании $G$.

Нормальные формы КС-грамматик:
\begin{itemize}
\item НФ Хомского ($G_Х$)
\item НФ Грейбах ($G_Г$)
\end{itemize}

\begin{defin}
  КС-грамматика задана в нормальной форме Хомского, если её правила
  имеют вид:
  
  $$
  \begin{cases}
    A \to BC\\
    A \to a,&\text{где $A,B,C \in N$, $a \in T$.}
  \end{cases}
  $$
  
  Будем обозначать такую грамматику $G_Х$.
\end{defin}

\begin{defin}
  КС-грамматика задана в нормальной форме Грейбах, если её правила
  имеют вид:
  
  $$
    A \to a\alpha,\;\text{где $A \in N$, $a \in T$, $\alpha \in \{(N
      \cup T)^* \cup \varnothing\}$.}
  $$
  
  Будем обозначать такую грамматику $G_Г$.
\end{defin}

\begin{theorem}
  Для любой КС-грамматики общего вида всегда существует хотя бы одна
  $G_Х$, а также $G_Г$:
  $$
  \xymatrix{
    & G \ar[ld] \ar[rd]\\
    G_Х \ar@{<->}[rr]&&G_Г
  }
  $$
\end{theorem}

\begin{rem}
  Эта теорема говорит о том, что изучение свойств любой КС-грамматики
  $G$ можно производить на эквивалентной ей $G_Х$ или $G_Г$, которые
  порождают тот же язык $L(G)$.
\end{rem}

Алгоритмы эквивалентных преобразований КС-грамматик приведены в [6]. и
желающие (в силу недостатка нашего времени) могут изучить их сами.

%%%%%%%%%%%%%%%%%%%%%%%%%%%%%%%%%%%%%%%%%%%%%%%%%%%%%%%%%%%%%%%%%%%%%%%%%%%%%%%%

\subsubsection{Анализ КС-грамматик}

Использование рассмотренных нормальных форм позволяет строить более
простые алгоритмы анализа КС-языков и грамматик.

Однако, как и в случае анализа автоматных грамматик, анализ
КС-грамматик может быть осуществлён двумя путями:
\begin{enumerate}
\item 
  $
  \xymatrix{
    G \ar[rr] & &
    {\left\{\begin{aligned}
      \alpha &\in L(G),\\
      \alpha &\notin L(G)
    \end{aligned}\right\}}\\ 
    \alpha \ar[u]
    }
  $
\item 
  $
  \xymatrix{
    G \ar[r] & G_{Х,Г} \ar[r] & S_{МП} \ar[r] & 
    {\left\{\begin{aligned}
      \alpha &\in L(G),\\
      \alpha &\notin L(G)
    \end{aligned}\right\}}\\ 
    &&\alpha \ar[u]
    }
  $
\end{enumerate}

1-й путь реализуется использованием большого количества алгоритмов:
\begin{itemize}
\item алгоритм на основе матрицы предшествования;
\item алгоритм Кнута;
\item алгоритм с использованием степенных рядов;
\item алгоритм Эрли;
\item \dots
\end{itemize}

2-й путь~--- с использованием $S_{МП}$.

%%%%%%%%%%%%%%%%%%%%%%%%%%%%%%%%%%%%%%%%%%%%%%%%%%%%%%%%%%%%%%%%%%%%%%%%%%%%%%%%

\subsubsection{Понятие об автомате с магазинной памятью}

МП-автомат (автомат с магазинной памятью является дальнейшим развитием~КА:
$$
\xymatrix{
  \begin{tabular}{c}
    $
    \overbrace{
      \begin{tabular}{|c|c|c|c|c|c|c|c|}
        \hline
        $x_1$ & $x_2$ & $x_3$ & \dots & $x_i$ & \dots & $x_{n-1}$ & $x_n$\\
        \hline
      \end{tabular}
    }^{\displaystyle\alpha}
    $
  \end{tabular}
  \\
  \fbox{\parbox[.7\height]{7em}{%
      \begin{center}
        Устройство управления МП-автоматом
      \end{center}
    }%
  } \ar@<2em>@{=>}[r] \ar@{=>}[u]
  &
  {\begin{tabular}[t]{|c|}
      \hline
      $z_i$\\\hline
      $z_{i-1}$\\\hline
      \dots\\\hline
      $z_0$\\\hline
    \end{tabular}}
  & \text{\parbox{6em}{
      \begin{center}
        стек (магазинная память)
      \end{center}
    }}
}
$$

\begin{defin}
  \emph{МП-автомат} определяется как \emph{семёрка}:
  \begin{equation*}
    S_{МП} = \langle X,Q,V,\delta,q_0,Q_F \rangle\,,
  \end{equation*}
\noindent где $X$~--- конечный алфавит входных символов МПА (как в КА);

$Q$~--- конечный алфавит внутренних состояний МПА (как в КА);

$q_0 \in Q$~--- начальное состояние;

$V$~--- конечное множество магазинных символов;

$z_0$~--- начальный символ магазина, находящийся всегда на дне;

$Q_F \subseteq Q$~--- множество заключительных (финальных) состояний
(как в КА);

\hangindent=4em$\sigma$~--- функция переходов:
\begin{equation*}
  \delta \colon Q \times X \times V \to \mathcal{B} (Q \times V^*)\,,
\end{equation*}
которая отображает тройки, образованные \emph{состоянием, входным
  символом, магазинным символом (доступным)~--- вершиной магазина}, в
множество всех подмножеств пар вида \emph{внутреннее состояние слова в
алфавите магазинных символов}.\\
$\mathcal{B}$~--- операция, называемая \emph{булеаном}.\\
\end{defin}

МП-автомат функционирует сменяя свои конфигурации.

\begin{defin}
  \emph{Конфигурация} МП-автомата~--- это элемент множества пар
  \begin{equation*}
    Q \times V^*=\{(q_i,\gamma)\}\,,
  \end{equation*}
  где $q_i \in Q_i$~--- текущее состояние МП-автомата;

  $\gamma \in V^*$ (слово в стеке магазинных символов)~--- текущее
  содержимое стека
\end{defin}

Смена конфигураций называется \emph{движением МП-автомата}.

МП-автомат используется как устройство, \emph{распознающее}
(допускающее или отвергающее) цепочки (стека), записанные на входной
ленте, просматривая символы этой цепочки слева направо \emph{без
  возвратов}.

\noindent Если:
\begin{enumerate}
\item автомат находится в конфигурации $(q,\gamma v)$ ($v$~--- верхний
  доступный для обозрения символ в стеке;
\item считывающая головка обозревает ячейку входной ленты с символом $a_i$\,;
\item $(q',\gamma\beta) \in \delta(q,x_i,v)$\,,
\end{enumerate}
то МП-автомат может изменить конфигурацию на $(q',\gamma')$, а символ
$x_i$ считается допущенным.

Для обозначения такого перехода используется запись:
\begin{equation*}
  x_i\colon (q,\gamma v) \to (q', \gamma\beta)
\end{equation*}

Если $x_i = \varepsilon$, то такой переход называется
\emph{$\varepsilon$-переходом}, т.\,е. автомат переходит из одной
конфигурации в другую \emph{без ввода входного символа}: входное слово
$\alpha = {x_i}_1,{x_i}_2,\ldots,{x_i}_n$ \emph{допускается
  МП-автоматом}, если, начиная с первого символа ${x_i}_1$ и начальной
конфигурации, автомат переходит после введения последнего символа
${x_i}_n$ в конфигурацию $(q_i,v_0\gamma)$, у которой $q_i \in Q_F$.

\begin{ex}
  Рассмотрим МП-автомат, допускающий цепочки языка $L = \{a^nb^n \mid
  n>1\}$.

  $L$~--- язык Дика: в программировании~--- язык описания скобочных
  структур.

  $L$~--- не может быть описан КА и автоматной грамматикой.
  
  $$S_{МП} = \langle X,Q,V,\delta,q_0,v_0,Q_F \rangle\,,$$
  где $X = \{a,b\}$, $Q = \{q_0,q_1,q_2,q_3\}$, $V = \{v\}$, $Q_F = \{q_3\}$;
  \begin{tabbing}
    $\delta(q_0,a,v_0) = \{(q_1,v)\}$\qquad\=$\delta(q_1,a,v) =
    \{(q_1,vv)\}$\=\qquad\=$\delta(q_1,b,vv) = \{(q_2,v)\}$\\
    $\delta(q_2,b,vv) = \{(q_2,v)\}$ \>
    $\delta(q_2,b,v_0v)=\{(q_3,v_0)\}$
  \end{tabbing}
  т.\,к. $\delta\colon Q \times X \times V \to B(Q \times V^*)\,.$

  $\alpha = a\ a\ a\ b\ b\ b$

  \begin{table}[h]
    \centering
    \begin{tabular}{c @{\qquad} c @{\qquad} c @{\qquad} l r}
    \hline\hline
    \textbf{№ шага} & $\mathbf{q}$ & $\mathbf{x_i}$ & \textbf{Стек} &
    \textbf{Оставшаяся часть слова} $\boldsymbol{\alpha}$ \\\hline
    0 & $q_0$ & $a$ & $v_0$ & $a\ a\ a\ b\ b\ b$\phantom{\qquad\qquad}\\
    1 & $q_1$ & $a$ & $v_0v$ & $a\ a\ b\ b\ b$\phantom{\qquad\qquad}\\
    2 & $q_1$ & $a$ & $v_0vv$ & $a\ b\ b\ b$\phantom{\qquad\qquad}\\
    3 & $q_1$ & $b$ & $v_0vvv$ & $b\ b\ b$\phantom{\qquad\qquad}\\
    4 & $q_2$ & $b$ & $v_0vv$ & $b\ b$\phantom{\qquad\qquad}\\
    5 & $q_2$ & $b$ & $v_0v$ & $b$\phantom{\qquad\qquad}\\
    6 & $q_3$ & $-$ & $v_0$\\\hline\hline
  \end{tabular}
\end{table}

$G\colon P=\{S \to aSb; S \to ab\}$
\end{ex}

\begin{theorem}
  Для любой КС грамматики $G$ существует такой МП автомат $S_{МП}$ и
  наоборот, что
  $$L(G) = L(S_{МП})\,.$$
\end{theorem}

%%%%%%%%%%%%%%%%%%%%%%%%%%%%%%%%%%%%%%%%%%%%%%%%%%%%%%%%%%%%%%%%%%%%%%%%%%%%%%%%

\subsubsection{Анализ КС-языков с использованием систем уравнений}

Как это отмечено ранее, существует большое многообразие алгоритмов
анализа КС-языков. Одним из них является алгоритм, основанный на
анализе степенных рядов.

\begin{defin}
  Существует $T = \{x_j \mid j \in I_T\}$~--- территориальный словарь
  некоторого языка $L(G)$, порождаемого КС-грамматикой $G$.

  Введем отображение $r$:
  $$\begin{cases}
    r \colon \alpha \to z & \text{или}\\
    r \colon T^* \to Z\,,
  \end{cases}$$
  которое каждому $\alpha \in T^*$ ставит в соответствие некоторый
  элемент $z \in Z$ (числовое множество).
\end{defin}

Тогда совокупность пар
$$
R = \sum_{\alpha \in L(G)} \langle z,\alpha \rangle
$$
назовём \emph{степенным рядом} языка $L(G)$ для КС-грамматики $G$, а
элементы $z \in Z$~--- \emph{коэффициентами} (или степенью)
\emph{соответствующего слова $\alpha$}.

В зависимости от состава множества $Z$ различают следующие степенные
ряды:
\begin{list}{\arabic{N})}{\usecounter{N}}
\item $Z = \{0,1\}$ $\Rightarrow$ $R$~--- называется
  характеристическим рядом;
\item $Z = \{0,1,2,3,\ldots\}$ $\Rightarrow$ $R$~--- ряд с целыми
  положительными коэффициентами, $z \in Z$~--- кратность слов;
\item $Z = \{[0,1]\}$ $\Rightarrow$ $R$~--- ряд с нормированными
  коэффициентами, $z$~--- вероятность слова $\alpha$.
\end{list}

Рассмотрим $Z = \{0,1\}$~--- характеристический степенной ряд, причём
при определении такого степенного ряда будем указывать только те члены
ряда, у которых $Z = 1$ ($Z \ne 0$).

Определим операции сложения и умножения на множестве степенных рядов.

\begin{defin}
  \emph{Сложением} двух рядов $R_1$ и $R_2$ назовём бинарную операцию,
  в результате выполнения которой формируется ряд $R$:
  $$R = R_1 + R_2$$
  такой, что коэффициенты $z$ в $R$ при каждой цепочке $\alpha$ есть
  сумма коэффициентов при аналогичных цепочках в рядах $R_1$ и $R_2$,
  причём:
  $$
  \begin{cases}
    \alpha + \alpha = 1\cdot\alpha + 1\cdot\alpha = (1 + 1)\cdot\alpha =
    1\cdot\alpha = \alpha\,;\\
    \alpha + 0 = 1\cdot\alpha + 0\cdot\alpha = (1 + 0)\cdot\alpha =
    1\cdot\alpha = \alpha\,.\\
  \end{cases}
  $$
  т.\,е. $z\alpha = z_1\alpha + z_2\alpha$, где $z = \max (z_1 +
  z_2)$, $\alpha \in T^*$.
\end{defin}

\begin{defin}
  \emph{Произведением} двух рядов $R_1$ и $R_2$ называют бинарную
  операцию, выполнение которой приводит к ряду $R$:
  $$R = R_1 \cdot R_2\,,$$
  такому, что коэффициенты $z$ формируются следующим образом:
  $$z = z_{i_1} z_{j_1} + z_{i_2} z_{j_2} + \ldots\,,$$
  для которых верно условие:
  $$\alpha_{i_1}\cdot\alpha_{j_1} = \alpha_{i_2}\cdot\alpha_{j_2} =
  \ldots = \alpha\,.$$
  где $\alpha_i\cdot\alpha_j$~--- операция конкатенации цепочек
  $\alpha_i$ и $\alpha_j$.
  
  Например, $ab \cdot bc = abbc$.
\end{defin}

Из рассмотренных определений вытекает следующая теорема:
\begin{theorem}
  Каждой КС-грамматике $G$ соответствует степенной ряд $R$.
\end{theorem}

Пусть $S, A, B \in N$~--- нетерминалы в $G$; $S$~--- аксиома.

Сопоставим каждому элементу из $N$ языковую переменную:
$$\begin{tabular}{c@{\;---\;}c}
  $S$ & $\gamma_1$\\
  $A$ & $\gamma_2$\\
  $B$ & $\gamma_3$\\
\end{tabular}$$

\begin{defin}
  Языковой переменной $\gamma$ назовём такую переменную, которая
  принимает значения на множестве $T^*$, т.\,е. $\gamma$~--- некоторый
  язык в алфавите $T$.
\end{defin}

Распределим правила $R$ грамматики $G$ на следующие группы:
\begin{list}{\arabic{N})}{\usecounter{N}}
\item $S \to \varphi_1$, $S \to \varphi_2$, \ldots, $S \to \varphi_k$\,,
\item $A \to \psi_1$, $A \to \psi_2$, \ldots, $A \to \psi_l$\,,
\item $B \to \theta_1$, $B \to \theta_2$, \ldots, $B \to \theta_m$\,.
\end{list}
где $\varphi_i$, $\psi_i$, $\theta_i$~--- слова в алфавите $(T \cup
N)^*$ (правые части правил).

Заменим в цепочках $\varphi_i$, $\psi_i$, $\theta_i$ каждое вхождение
нетерминального символа соответствующей переменной, а замет объединим\\
все цепочки $\varphi$ в формальное выражение $f$:
$$f = \varphi_1 + \varphi_2 + \ldots + \varphi_k\,,$$
все цепочки $\varphi$ в формальное выражение $g$:
$$g = \psi_1 + \psi_2 + \ldots + \psi_l\,,$$
все цепочки $\varphi$ в формальное выражение $h$:
$$f = \theta_1 + \theta_2 + \ldots + \theta_m\,,$$
и т.\,д.

Тогда получим систему уравнений:
$$
\begin{cases}
  \gamma_1 = f(\gamma_1, \gamma_2, \gamma_3, \ldots)\,;\\
  \gamma_2 = g(\gamma_1, \gamma_2, \gamma_3, \ldots)\,;\\
  \gamma_3 = h(\gamma_1, \gamma_2, \gamma_3, \ldots)\,;\\
  \ldots
\end{cases}
$$

Решение полученной системы уравнений с целью определения значений
языковых переменных $\gamma_1, \gamma_2, \gamma_3, \ldots$
осуществляется методом последовательных приближений.

На начальном (нулевом) шаге языковым переменным присваиваются значения
$0$.
$$
\begin{cases}
  \gamma_1^{(0)} = 0\,,\\
  \gamma_2^{(0)} = 0\,,\\
  \gamma_3^{(0)} = 0\,,\\
  \ldots\\
\end{cases}
$$

На последующих шагах в правые части уравнений вместо языковых
переменных $\gamma_1, \gamma_2, \gamma_3, \ldots$, входящих в состав
\emph{степенных рядов} $f, g, h, \ldots$, подставляются их значения с
предыдущего шага, т.\,е. формально на каждом $k$-ом шаге решение
уравнений имеет вид:
$$
\begin{cases}
  \gamma_1^{(k)} = f(\gamma_1^{(k-1)}, \gamma_2^{(k-1)},
  \gamma_3^{(k-1)}, \ldots)\,;\\
  \gamma_2^{(k)} = g(\gamma_1^{(k-1)}, \gamma_2^{(k-1)},
  \gamma_3^{(k-1)}, \ldots)\,;\\ 
  \gamma_3^{(k)} = h(\gamma_1^{(k-1)}, \gamma_2^{(k-1)},
  \gamma_3^{(k-1)}, \ldots)\,;\\ 
  \ldots
\end{cases}
$$

В общем случае, исходя из начального приближения, будем
последовательно получать значения языковых переменных, а в качестве
индуктивного сформированного предельного решения будем иметь
совокупность таких значений языковых переменных:
$$
\begin{cases}
  \gamma_1 = \gamma_1^{(1)} + \gamma_1^{(2)} + \gamma_1^{(3)} + \ldots
  + \gamma_1^{(k)} + \ldots ;\\
  \gamma_2 = \gamma_2^{(1)} + \gamma_2^{(2)} + \gamma_2^{(3)} + \ldots
  + \gamma_2^{(k)} + \ldots ;\\
  \gamma_3 = \gamma_3^{(1)} + \gamma_3^{(2)} + \gamma_3^{(3)} + \ldots
  + \gamma_3^{(k)} + \ldots ;\\
  \ldots\\
\end{cases}
$$

Отметим, что значение каждой языковой переменной, представляющей
степенной ряд, образует, соответственно, язык:
$$
\begin{cases}
  \gamma_1 = L_S\,,\\
  \gamma_2 = L_A\,,\\
  \gamma_3 = L_B\,,\\
  \ldots\\
\end{cases}
$$
Язык при аксиоме $L_S$ есть $L_S = L(G)$.

Поскольку результатом анализа цепочки $\alpha$ по заданной грамматике
$G$ является вывод о принадлежности $\alpha$ языку $L(G)$,
порождаемому грамматикой $G$, то процесс анализа реализуется следующим
образом.

Мы имеем входное слово $\alpha$ длины $n$ на каждом $i$-ом шаге
приближения среди пар $\langle z_j, \alpha_j \rangle$ из состава
степенного ряда при аксиоме $S$ ищется та пара, у которой
$$
\begin{cases}
  z_j = 1\,,\\
  \alpha_j = \alpha\,.\\
\end{cases}
$$

Если такой пары нет, то ищется решение на следующем шаге приближения и
т.\,д.

Такие шаги делаются до тех пор, пока:
\begin{itemize}
\item не будет найден элемент $\langle z_j, \alpha_j \rangle$,
  $\alpha_j = \alpha$ в значении $\gamma_1$ (при аксиоме) ($\alpha \in L(G)$);
\item не будет произведено такое максимально необходимое
  (гарантированное) число шагов $R_{max}$, что все возможные цепочки
  $\alpha_j$ длины $n$ уже порождены, но ни одна из них не совпадает с
  $\alpha$ ($\alpha \in L(G)$).
\end{itemize}

\begin{ex}
  Дана грамматика $G$:
  \begin{center}
    \begin{tabular}{l}
      $S \to aB \mid bA$,\\
      $A \to aS \mid bAA \mid a$,\\
      $B \to bS \mid aBB \mid b$.\\
    \end{tabular}
  \end{center}
  $L(G)$~--- язык, включающий все те и только те слова, которые
  состоят из равного числа символов $a$ и $b$ (например, $\alpha =
  bbaa$, $\alpha = abba$).
\end{ex}


%%% Local Variables: 
%%% mode: latex
%%% TeX-master: "lections"
%%% End: 
